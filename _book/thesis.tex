%%%---PREAMBLE---%%%%%%%%%%%%%%%%%%%%%%%%%%%%
\documentclass[twoside,12pt,final]{ucthesis-CA2012} % change this for oneside, 11pt, draft, etc.

% fix for pandoc 1.14
\providecommand{\tightlist}{%
  \setlength{\itemsep}{0pt}\setlength{\parskip}{0pt}}

%--- Packages --------------------------------------------------------
\usepackage[lofdepth,lotdepth,caption=false]{subfig}
\usepackage{fancyhdr}
\usepackage{amsmath, amssymb, graphicx}
\usepackage{xspace}
\usepackage{braket}
\usepackage{color}
\usepackage{setspace}
\usepackage{fancyvrb}
\usepackage{array}
\usepackage{ifxetex,ifluatex}
\usepackage{etoolbox}

%% for the per mil symbol
\usepackage[nointegrals]{wasysym}

%% for more attractive tables
\usepackage{booktabs}
\usepackage{xcolor}
\usepackage{longtable}
\usepackage{lscape}
\usepackage{tabularx}

\usepackage[nostamp]{draftwatermark}
% % Use the following to make modification
\SetWatermarkText{DRAFT}
\SetWatermarkLightness{0.95}

% suppress bottom page numbers on 1st page of each chapt.
% because they overlap with text
%\patchcmd{\chapter}{plain}{empty}{}{} % turn off for UCD requirements of page number on every page

%---New Definitions and Commands------------------------------------------------------

\newtheorem{theorem}{Jibberish}

\bibliography{references}

\hyphenation{mar-gin-al-ia}

% from uw_template.tex

% commands and environments needed by pandoc snippets
% extracted from the output of `pandoc -s`
%% Make R markdown code chunks work

\ifxetex
  \usepackage{fontspec,xltxtra,xunicode}
  \defaultfontfeatures{Mapping=tex-text,Scale=MatchLowercase}
\else
  \ifluatex
    \usepackage{fontspec}
    \defaultfontfeatures{Mapping=tex-text,Scale=MatchLowercase}
  \else
    \usepackage[utf8]{inputenc}
  \fi
\fi
\DefineShortVerb[commandchars=\\\{\}]{\|}
\DefineVerbatimEnvironment{Highlighting}{Verbatim}{commandchars=\\\{\}}
% Add ',fontsize=\small' for more characters per line
\newenvironment{Shaded}{}{}
\newcommand{\KeywordTok}[1]{\textcolor[rgb]{0.00,0.44,0.13}{\textbf{{#1}}}}
\newcommand{\DataTypeTok}[1]{\textcolor[rgb]{0.56,0.13,0.00}{{#1}}}
\newcommand{\DecValTok}[1]{\textcolor[rgb]{0.25,0.63,0.44}{{#1}}}
\newcommand{\BaseNTok}[1]{\textcolor[rgb]{0.25,0.63,0.44}{{#1}}}
\newcommand{\FloatTok}[1]{\textcolor[rgb]{0.25,0.63,0.44}{{#1}}}
\newcommand{\CharTok}[1]{\textcolor[rgb]{0.25,0.44,0.63}{{#1}}}
\newcommand{\StringTok}[1]{\textcolor[rgb]{0.25,0.44,0.63}{{#1}}}
\newcommand{\CommentTok}[1]{\textcolor[rgb]{0.38,0.63,0.69}{\textit{{#1}}}}
\newcommand{\OtherTok}[1]{\textcolor[rgb]{0.00,0.44,0.13}{{#1}}}
\newcommand{\AlertTok}[1]{\textcolor[rgb]{1.00,0.00,0.00}{\textbf{{#1}}}}
\newcommand{\FunctionTok}[1]{\textcolor[rgb]{0.02,0.16,0.49}{{#1}}}
\newcommand{\RegionMarkerTok}[1]{{#1}}
\newcommand{\ErrorTok}[1]{\textcolor[rgb]{1.00,0.00,0.00}{\textbf{{#1}}}}
\newcommand{\NormalTok}[1]{{#1}}
\newcommand{\OperatorTok}[1]{\textcolor[rgb]{0.00,0.44,0.13}{\textbf{{#1}}}}
\newcommand{\BuiltInTok}[1]{\textcolor[rgb]{0.00,0.44,0.13}{\textbf{{#1}}}}
\newcommand{\ControlFlowTok}[1]{\textcolor[rgb]{0.00,0.44,0.13}{\textbf{{#1}}}}

\ifxetex
  \usepackage[setpagesize=false, % page size defined by xetex
              unicode=false, % unicode breaks when used with xetex
              xetex,
              colorlinks=true,
              linkcolor=blue]{hyperref}
\else
  \usepackage[unicode=true,
              colorlinks=true,
              linkcolor=blue]{hyperref}
\fi
\hypersetup{breaklinks=true, pdfborder={0 0 0}}
\setlength{\parindent}{0pt}
\setlength{\parskip}{6pt plus 2pt minus 1pt}
\setlength{\emergencystretch}{3em}  % prevent overfull lines
\setcounter{secnumdepth}{0}

%---Set Margins ------------------------------------------------------
\setlength\oddsidemargin{0.25 in} \setlength\evensidemargin{0.25 in} \setlength\textwidth{6.25 in} \setlength\textheight{8.5 in} %8.5
\setlength\footskip{0.25 in} \setlength\topmargin{0 in} \setlength\headheight{0.25 in} \setlength\headsep{0.25 in}

%%%---DOCUMENT---%%%%%%%%%%%%%%%%%%%%%%%%%%%%
\begin{document}

%=== Preliminary Pages ============================================
\begin{ucfrontmatter}

  %%%%%%%%%%%%%%%%%%%%%%%%%%%
  % TITLE PAGE INFORMATION % % modified to meet UCDavis
  %%%%%%%%%%%%%%%%%%%%%%%%%%%

  \title{Population genetics of a sentinel stream-breeding frog (\emph{Rana
boylii})}

  \author{Ryan A Peek}
  \prevdegreeA{B.S. (University of California, Davis) 2002}
  \prevdegreeB{M.S. (University of San Francisco) 2010}
  \report{DISSERTATION} 
  \degree{DOCTOR OF PHILOSOPHY} 
  \degreemonth{September} \degreeyear{2018}
  \chair{Michael R. Miller}  % this is your advisor
  \othermemberA{Peter B. Moyle} % This is a member of your committee
  \othermemberB{Mark W. Schwartz} % This is a member of your committee
  \othermemberC{} % This is a member of your committee
  \numberofmembers{3} % should match the number of entries above (chair + othermembers)

  \field{Ecology}
  \campus{Davis}
	
	\maketitle
	\approvalpage
	%\copyrightpage % if you want

  % DEDICATION %
  %%%%%%%%%%%%%%%%%%%%
    \begin{dedication}

      \vspace*{25ex}
      \begin{center}
      \begin{Large}

        ``\emph{One thing to remember is to talk to the animals. If you do, they
        will talk back to you. But if you don't talk to them, they won't talk
        back to you, then you won't understand. And when you don't understand,
        you will fear, and when you fear, you will destroy the animals, and if
        you destroy the animals, you will destroy yourself}''\\
        (Chief Dan George, Tseil-Waututh Nation, North Vancouver)

      \end{Large}
      \end{center}
  \end{dedication}
  % ACKNOWLEDGEMENTS %
  %%%%%%%%%%%%%%%%%%%%
  \begin{acknowledgements}
    For all the curious people who have come before and hopefully after, I
    want to acknowledge you, and I hope we can do better to inspire and
    support those voices that may not have had the opportunities or
    priviledge I have had. I am lucky to have had all I have had, and
    finishing a dissertation requires a community, and this dissertation
    would not have happened if it wasn't for the amazing community of
    family, friends, and colleagues who helped me every step of the way. In
    particular, thank you to my partner, wife and best-friend Leslie---you
    are my sun and gravity---you held me together, anchored our family, and
    made it possible to run this crazy academic ultra-marathon. To my
    dearest little tadpoles, Connor and Genevieve, you inspire me, you make
    me laugh every day, and you remind me the world still has hope as long
    as we nourish joy and curiosity. Thank you for being you, and I hope one
    day you forgive me for the amount of time I've spent staring at a
    computer. Thanks to my mom for all the support, love, and baked goods.
    Sibling, thank you for consistently inspiring me, listening to me, and
    being the best sibling one could ask for. And for all my close friends,
    bandmates, and officemates (you know who you are), you keep me sane, you
    motivate me, and you remind me every day that I really love this crazy
    journey. John, your shed and couch have probably single-handedly kept me
    anchored in ways I can't even express\ldots{}also your friendship. Thank
    you for your time, humor, and general levity. Thanks to my Dad, who has
    cajoled, pestered, and annoyed me for far too long to ``get a PhD'',
    thanks for believing it was possible even when I didn't. Also, please
    never suggest anything like this again. And to my committee and my
    colleagues at the Center for Watershed Sciences, you have all been an
    amazing resource in providing feedback, guidance, and support throughout
    my graduate student career. Finally, to my cohort and fellow students in
    the GGE, this has been a great place to grow and mature as a scientist
    and researcher. Thank you all.
  \end{acknowledgements}
  % removed CV section from this but see gauchodown or huskydown

  % ABSTRACT %
  %%%%%%%%%%%%%%%%%%%%%%%%%%%
  \begin{abstract}
    \addcontentsline{toc}{chapter}{Abstract}

    \emph{Rana boylii} is an imperiled frog species native to CA and OR, and
    it is currently designated as a species of special concern (CDFW) in the
    state of CA. It has been petitioned as candidate for federal (USFWS) and
    state (CDFW) listing. As a lotic breeding amphibian, \emph{R. boylii} is
    tied closely to local flow regimes in the watersheds it inhabits and is
    therefore particularly sensitive to alterations to the natural flow
    regime. Effective conservation management of this species should
    consider and prioritize maintenance of genetic diversity as part of any
    listing decision because it is closely related to the evolutionary
    capacity for adaptation to environmental changes. Conservation of
    genetic diversity in this species will require several components,
    including refining potential conservation units (i.e., distinct
    population segments) and quantifying of genetic diversity and genetic
    diversity trajectories across the species range. To assess these
    components, fine-scale and landscape-scale analyses were conducted using
    genomic data from over 600 samples from 89 localities across the range
    of the species. Six genomically-distinct groups were identified, as well
    as population subdivisions at local watershed scales. One major impact
    on \emph{R. boylii} populations has been river regulation. River
    regulation has been implicated as a cause of fundamental changes to
    downstream aquatic ecosystems. Regulation changes the natural flow
    regime which may restrict population connectivity and decrease genetic
    diversity in some species. Since population connectivity and the
    maintenance of genetic diversity are fundamental drivers of long-term
    persistence, understanding the extent that river regulation impacts
    these critical attributes of genetic health is an important goal.
    However, the extent to which \emph{R. boylii} populations in regulated
    rivers have maintained connectivity and genetic diversity is unknown.
    The impacts of river regulation on \emph{R. boylii} were investigated
    with genomic data to explore the potential for long-term persistence of
    \emph{R. boylii} under continued regulation. \emph{R. boylii} in
    regulated rivers showed striking patterns of isolation and trajectories
    of genetic diversity loss relative to unregulated rivers. For example,
    river regulation explained the greatest amount of variance in population
    genetic differentiation compared with other covariates including
    geographic distance. Importantly, patterns of connectivity and genetic
    diversity loss were observed regardless of regulation level but were
    most prominent in locations with the greatest regulation intensity.
    Using the same genomic data, fine-scale analyses of \emph{R. boylii} and
    \emph{R. sierrae} in a single region of the Sierra Nevada of California
    was conducted to evaluate the potential for hybridization between
    species. Hybridization between species may combine parental genotypes in
    ways that yield reproductively sterile or isolated lineages, and
    hybridization events may be short-lived and difficult to detect. Limited
    hybridization between the species was detected in the Feather basin,
    though it appears these are terminal events based on PCA, admixture, and
    tests of heterozygosity using species diagnostic SNPs. Finally,
    rangewide quantification and comparison of genomic variation across
    populations indicates the southern coast, southern Sierra Nevada, and
    Northern Sierra/Feather basin in California should have high
    prioritization in conservation efforts due to low genomic diversity and
    trajectories of diversity loss. More broadly, these results demonstrate
    both the critical need for regional conservation in a sentinel river
    species, and the utility and power of genetic methods for assessing and
    monitoring sensitive species across many scales.

    %\abstractsignature
  \end{abstract}
  % Table of Contents %
  %%%%%%%%%%%%%%%%%%%%%%%%%%%
	\tableofcontents

\end{ucfrontmatter} % end of the preliminary pages
\begin{ucmainmatter}

\fancypagestyle{plain}{%
  \renewcommand{\headrulewidth}{0pt}%
  \fancyhf{}%
  \fancyfoot[C]{\thepage}
  \setlength\footskip{28pt}
}
\pagestyle{plain}

\hypertarget{reg-health}{%
\chapter{Flow regulation associated with decreased genetic health of a
river-breeding frog species}\label{reg-health}}

\hypertarget{introduction}{%
\section{Introduction}\label{introduction}}

Rivers simultaneously connect and carve the landscapes through which
they flow. Rivers provide corridors of connectivity for riparian and
aquatic organisms such as fish, amphibians, and macroinvertebrates
(Wiens 2002, Pringle 2003), while also acting as physical barriers on
the landscape for many terrestrial organisms (Voelker et al. 2013, Cazé
et al. 2016). Hydrologic connectivity (Pringle 2003) transfers energy,
organisms and ultimately genetic variation and thus is a critical
component for population persistence in dynamic systems where
populations must constantly adapt to temporal and spatial changes. In
Mediterranean climates, rivers have strong seasonal patterns associated
with cold, wet winters and warm, dry summers. Native aquatic organisms
have evolved life histories well adapted to these natural patterns,
which are both predictable and seasonal (Yarnell et al. 2010, Tonkin et
al. 2017).

River regulation, or the hydrological alteration of flow by dams and
diversions, impacts the seasonal and interannual flow variability within
a watershed. Regulation changes the natural flow regime and dramatically
alters geomorphic and hydrologic connectivity of watersheds (Poff et al.
2007), which may restrict natural population connectivity (Schick and
Lindley 2007, Shaw et al. 2016). River regulation can change flow
frequency, magnitude, duration, timing, and rate of change, which can
have significant impacts on aquatic organisms and ecological processes
(Poff et al. 2007, Yarnell et al. 2010). River regulation, and more
specifically, regulation associated with hydropower generation, has been
implicated as a cause of fundamental changes to downstream aquatic
ecosystems (Power et al. 1996, Bunn and Arthington 2002, Moyle et al.
2011). The hydrological regimes of over half of the world's largest
rivers have been altered by large dams (Nilsson et al. 2005) and only
recently has the extent of flow alteration and the associated
ecosystem-level impacts been acknowledged (Pringle 2001, Dudgeon et al.
2006, Murchie et al. 2008).

Changes to abiotic processes caused by river regulation can have a
substantial impact on biotic communities. The negative effects of river
regulation on migration and loss of spawning habitat (Fuller, Pope,
Ashton, \& Welsh, 2011; Sarah J. Kupferberg et al., 2012; Lind, Welsh,
\& Wilson, 1996; Rolls \& Bond, 2017), reductions in population
abundances and diversity (Fuller et al., 2011; Guzy, Eskew, Halstead, \&
Price, 2018; Lind et al., 1996; Sabo et al., 2017; Scribner et al.,
2016; Vörösmarty et al., 2010; Zhong \& Power, 1996), and fragmentation
(Guzy et al., 2018; Sabo et al., 2017; Scribner et al., 2016; Vörösmarty
et al., 2010; Werth, Schödl, \& Scheidegger, 2014; Zhong \& Power, 1996)
have been well documented. However, most rivers have not been regulated
for long periods (e.g., less than 100 years) compared to the time these
organisms had to adapt to pre-anthropogenic river flow. In regulated
rivers that organisms still occupy, it remains unknown whether
populations can persist long-term with continued regulation. In other
words, while some species may have persisted since regulation began in a
system (e.g., several decades), this does not necessarily mean these
populations will persist into the future under current flow regulation
regimes. Thus, exploring the potential for long-term persistence of
populations under different flow regimes is a crucial component for
guiding conservation efforts yet remains a significant gap.

One tool that can help address this gap is the integration of genetics
and hydrology to better assess the impact of river regulation on aquatic
organisms (Scribner et al., 2016). Although aquatic organisms are often
difficult to count and monitor by conventional methods, genetic
monitoring can be a powerful tool to assess population health by
revealing factors such as fragmentation and population declines. It is
widely recognized that reductions in population connectivity can
increase isolation and inbreeding, leading to a potential ``extinction
vortex'' (Gilpin \& Soule, 1986), yet there is limited understanding of
how flow alteration may impair the processes crucial for maintenance of
genetic variation and thus adaptive capacity. In addition, there is a
current pressing need for more effective and flexible watershed
management tools, particularly in relation to monitoring aquatic
populations and implementation of environmental flows (Grantham,
Merenlender, \& Resh, 2010). Thus, population genetics could be a
powerful tool to understand the influence of different flow regimes on
population health and this information could facilitate improved flow
management to better protect aquatic populations.

The river-breeding foothill yellow-legged frog (Rana boylii; FYLF)
historically occurred in lower and mid-elevation streams and rivers from
Southern Oregon to northern Baja California west of the Sierra-Cascade
crest (Stebbins, 2003). FYLF are intimately linked with river hydrology
because they have evolved to spawn in synchrony with natural flow cues
associated with seasonal spring snowmelt or rain recession periods
(Bondi, Yarnell, Lind, \& Lind, 2013; S. J. Kupferberg, 1996; S. M.
Yarnell et al., 2010; S. Yarnell, Peek, Epke, \& Lind, 2016). However,
population declines have been documented across the former range of this
species, particularly in southern California and the Sierra Nevada where
it has been extirpated from approximately 50 percent of its historical
range (Davidson, Shaffer, \& Jennings, 2002; Jennings \& Hayes, 1994).
In California, particularly in the Sierra Nevada, river regulation may
be a significant environmental stressor (Sarah J. Kupferberg et al.,
2012; Lind et al., 1996). Regulated river reaches typically alter flows
by augmenting or diverting winter and spring runoff, thereby reducing or
eliminating flow cues and disrupting natural flow regimes. Aseasonal
flow fluctuation from river regulation can scour (detach from substrate)
or desiccate FYLF egg masses, and the loss of clutches may have a
significant demographic impact because only one egg mass is laid per
year. In many regulated rivers in the Sierra Nevada, FYLF populations
are now restricted to small unregulated tributaries flowing into the
regulated mainstem.

Here, we investigate the impacts of river regulation on genetic health
of FYLF populations across three different flow regimes. Given that
population connectivity and genetic diversity are known to be play
critical roles in long-term species persistence, we explore the
association between these metrics and levels of river regulation. Our
goal is to assess the genetic health of FYLF under different river
regulation regimes to better inform the potential for long-term
persistence. Addressing this question will help to inform management and
conservation efforts for FYLF, as well as the potential utility of
genetics for future conservation monitoring efforts in aquatic species.

\hypertarget{methods}{%
\section{Methods}\label{methods}}

\hypertarget{sample-collection}{%
\subsection{Sample collection}\label{sample-collection}}

345 FYLF buccal or tissue samples were used in this study (see Table
S1). Field sampling was conducted as previously described (Heyer,
Donnelly, McDiarmid, Hayek, \& Foster, 1994), under CDFW SCP Permit
\#0006881, with IACUC protocol \#19327. Individual post-metamorphic
frogs were buccal-swabbed following established protocols (Broquet,
Berset-Braendli, Emaresi, \& Fumagalli, 2007; Goldberg, Kaplan, \&
Schwalbe, 2003; Pidancier, Miquel, \& Miaud, 2003). Each
post-metamorphic individual was comprehensively swabbed underneath
tongue and cheek for approximately one minute. Swabs were air dried for
approximately five minutes and placed in 1.5 mL microcentrifuge tubes
while in the field. Samples were stored in the laboratory at -80°C until
DNA extraction. Where possible, tail clips from tadpole larvae were
collected, and tadpoles greater than 15 mm total length were targeted
(Parris et al., 2010; Wilbur \& Semlitsch, 1990). One small
(\textless{}3mm) tail clip was taken per individual tadpole and dried on
Whatman qualitative filter paper (grade 1) and stored at room
temperature.

\hypertarget{de-novo-assembly}{%
\subsection{De novo assembly}\label{de-novo-assembly}}

To produce a high-quality genomic resource for a frog species with a
large genome size, we first interrogated a large fraction of the genome
using RAD sequencing (Baird et al., 2008; Miller et al., 2007).
Paired-end sequence data were generated from 24 FYLF individuals
(sampling details given in Table S2) across coastal and Sierra Nevada
populations from California, USA. DNA was extracted with a magnetic
bead--based protocol (Ali et al., 2016) and quantified using Quant-iT
PicoGreen dsDNA Reagent (Thermo Fisher Scientific) with an FLx800
Fluorescence Reader (BioTek Instruments). RAD libraries were constructed
using the SbfI restriction enzyme and a new RAD protocol (Ali et al.,
2016). De novo loci discovery and contig extension were carried as
previously described (Miller et al., 2012) using the alignment program
Novoalign and the genome assembler PRICE (Ruby, Bellare, \& Derisi,
2013). This pipeline resulted in a set of 77,544 RAD contigs ranging
from 300 to 800 bp (Table S3) which served as a de novo partial genome
reference for all subsequent downstream analyses.

\hypertarget{rapture-sequencing}{%
\subsection{Rapture sequencing}\label{rapture-sequencing}}

We then performed Rapture on all samples (Table S1) (2016) using 8,533
RAD capture baits (120 bp) were designed by Arbor Biosciences from the
de novo alignment (Table S4). The final Rapture library was sequenced in
50\% of an Illumina HiSeq 3000 lane. Rapture sequence data from each
individual (Table S1) were aligned against the de novo partial genome
reference using the BWA-MEM algorithm (Li, 2013; Li \& Durbin, 2010) and
saved to BAM format. SAMtools was used to sort, filter for proper pairs,
remove PCR duplicates, and index binary alignment map (BAM), as well as
merge sequences from multiple libraries (Li et al., 2009). BAM files
from the same sample were merged before indexing using SAMtools.

\hypertarget{principal-component-analysis}{%
\subsection{Principal component
analysis}\label{principal-component-analysis}}

A probabilistic framework was used to discover SNPs for PCA as it does
not require calling genotypes and is suitable for low-coverage
sequencing data (Fumagalli et al., 2013; Korneliussen, Moltke,
Albrechtsen, \& Nielsen, 2013). All Rapture analyses were conducted
using Analysis of Next Generation Sequencing Data (ANGSD) (Korneliussen,
Albrechtsen, \& Nielsen, 2014). ANGSD analyses were conducted following
methods from Prince et al (2017), with a minimum mapping quality score
(minMapQ) of 10, a minimum base quality score (minQ) of 20, and the
genotype likelihood model (GL 1) (Li, 2011). To maximize data quality,
samples with less than 100,000 aligned reads were excluded (Table S1,
S2) using and only sites represented in at least 50\% of the included
samples (minInd) were used. Settings used in ANGSD for PCA to identify
polymorphic sites included a SNP\_pval of 1e-6, inferring major and
minor alleles (doMajorMinor 1), estimating allele frequencies (doMaf 2)
(Kim et al., 2011), retaining SNPs with a minor allele frequency of at
least 0.05 (minMaf), genotype posterior probabilities were calculated
with a uniform prior (doPost 2), and the doIBS 1 and doCov 1 options
were used to generate PCA data. Principal components (PC) summarizing
population structure were derived from classic eigenvalue decomposition
and were visualized using the ggplot2 package in R (R Core Team, 2017).

\hypertarget{genetic-differentiation-and-diversity-estimates}{%
\subsection{Genetic differentiation and diversity
estimates}\label{genetic-differentiation-and-diversity-estimates}}

Mean scaled FST was used to quantify genetic differentiation between
populations (Rousset, 1997; Wright, 1943). Genome-wide FST between
population pairs was estimated by first calculating a site frequency
spectrum (SFS) for each population (doSaf) (Nielsen, Korneliussen,
Albrechtsen, Li, \& Wang, 2012) with ANGSD. The two-dimensional SFS and
global FST between each population pair were then estimated using
realSFS (Korneliussen et al., 2014). FST was calculated between each
pair of collection locations within a watershed, and the mean of all
pairwise calculations within that watershed was calculated for each
location. We calculated the river distances (distance along river
network) between locations within watersheds using the riverdist package
in R (Tyers, 2017), and used the mean pairwise river distance (km) to
all other locations within the watershed. These values were plotted and
a generalized linear model was fitted (FST \textasciitilde{} Mean River
Distance) in R (R Core Team, 2017). To calculate Watterson's θS
(Watterson, 1975), and Tajima's θ𝜋 (Tajima, 1983), we used SFS that were
estimated as described above as priors (pest) to calculate each
statistic for each site (doThetas), which were averaged to obtain a
single value for each statistic (Korneliussen et al., 2013).

\hypertarget{boosted-regression-tree-modeling-of-variance-in-fst}{%
\subsection{Boosted regression tree modeling of variance in
FST}\label{boosted-regression-tree-modeling-of-variance-in-fst}}

We used boosted regression tree (BRT) models with the R packages gbm
(Ridgeway, 2015) and dismo (Hijmans, Phillips, Leathwick, \& Elith,
2017) to assess the relative influence of river regulation as compared
to other covariates. Boosted regression trees (BRT) are suitable
frameworks for large and complex ecological datasets because they do not
assume normality, nor linear relationships between predictor and
response variables and they ignore non-informative predictor variables
(Graham et al., 2008; Steel, Peek, Lusardi, \& Yarnell, 2017). BRTs use
iterative boosting algorithms to combine simple decision trees to
improve model performance (De'ath, 2007) and provide a robust
alternative to many traditional statistical methods (Guisan et al.,
2007; Phillips, Anderson, \& Schapire, 2006). BRTs assess the relative
impact of modeled variables by calculating the number of times a
variable is selected for splitting a tree across all folds of the cross
validation. Following Steel et al.~2017, estimates of relative influence
for each predictor variable were used to evaluate the relative
contribution a variable had in predicting the response. To evaluate the
relative influence of covariates on FST, models were trained using river
distance (km), elevation (m), upstream drainage area (km2), Strahler
stream order, and number of samples per location. Stream segment data on
elevation, length, slope, stream order, and drainage area were derived
from NHD Plus attributes (U.S. Geological Survey, National Hydrography
Dataset, Digital data, accessed, August 2017 at
\url{http://nhd.usgs.gov/data.html}). In addition, Δθ (θ𝜋 − θS) was
included to assess the effect of genomic variation on FST across
regulation types.

Model training and fitting were conducted following methods previously
described in (Steel et al., 2017). To reduce overfitting, the learning
rate (also known as the shrinking rate) was set to 0.001. Stochastic
gradient boosting was utilized to reduce prediction error (De'ath, 2007)
and the fraction of training data sampled to build each tree was 0.75,
within the range as recommended by (Brown et al., 2012). Tree complexity
was set to three to allow for second and third order interaction
effects. The minimum number of observations required in the final nodes
of each tree was three. A ten-fold cross-validation technique allowed us
to determine the number of trees at which prediction error was minimized
using the cross-validation deviance. Model performance was evaluated
using the minimum estimated cross-validation deviance which maximized
the estimated deviance explained.

\clearpage

\hypertarget{results}{%
\section{Results}\label{results}}

\textbf{\emph{Rapture produces high quality genomic data for FYLF}}

To begin investigating the impact of river regulation on FYLF, we
collected frog tissue and buccal samples from 30 locations in six rivers
representing three different flow impairment levels associated with
hydropower generation. The three flow regimes assessed were: 1)
hydropeaking, where flows are pulsed on most days from late spring
through fall to provide electricity during peak-use hours and for
recreational whitewater rafting; 2) bypass, which diverts river flows
from an upstream portion of the basin to the downstream power generation
facilities; and 3) unregulated, a largely natural flow regime where no
upstream controls exist to regulate flows (Figure 1). Flow data were
obtained for each river reach using proximal USGS gaging stations (Table
S5). We sampled a total of 345 FYLF from sites in three major watersheds
(Yuba, Bear, and American) in the northern Sierra Nevada of California
(Figure 1A; Table 1). The six study rivers share a similar Mediterranean
climate, underlying geology, watershed aspect (west-slope), stream
morphology (riffle-pool), and vegetative communities, but differ in the
intensity of flow regulation (Steel et al., 2017). Although river
regulation occurs in all three of the study watersheds, both the North
Yuba and North Fork (NF) American are unregulated whereas the Middle
Fork (MF) American is the only river that has a hydropeaking flow regime
(Figure 1A).

To generate genetic data from the samples, we performed RAD Capture
(a.k.a. Rapture) (Ali et al., 2016) on the samples by generating SbfI
RAD libraries, capturing a subset of the RAD loci using 8,533 baits (see
Methods), and sequencing the resulting library on an Illumina HiSeq. We
then aligned the sequencing reads from each sample to a de novo RAD
assembly (see Methods). The mean number of filtered alignments across
all 345 samples was 324,928. For downstream analysis, we selected
individuals that had greater than 100,000 alignments (n=277), which
provided sufficient data to investigate population genetic attributes at
broad and fine geographic scales (see below). FYLF are cryptic, and
often occur in low densities within the study area. Thus, we retained a
minimum of three individuals per site, and the mean number of samples
per site was approximately nine (Table 1). With genomic data, population
genetic parameters can be accurately estimated from even low sample
numbers (Hotaling et al., 2018), and genomic analyses in non-model
organism often use fewer loci (Narum, Buerkle, Davey, Miller, \&
Hohenlohe, 2013). We conclude that the sequence data we obtained should
be appropriate for population genetic analyses across our study area.

\textbf{\emph{Anomalous genetic pattern in highly regulated reach of
Middle Fork American watershed}}

To assess FYLF population structure across the collection locations, we
used ANGSD (Korneliussen et al., 2014) to discover 44,406 SNPs and
perform principal component analysis (PCA; see Methods), which provides
a dimensionless comparison of all samples. The first two principal
components revealed four main groups corresponding to the Yuba, Bear,
North Fork (NF) American, and Middle Fork (MF) American samples (Figure
2A). Unlike the Yuba watershed where all rivers clustered as one group,
the two rivers within the American watershed (the NF American and MF
American) were separated by both PC1 and PC2. Although the NF American
watershed clustered closely with the adjacent Bear watershed, the MF
American showed a surprisingly high degree of genetic differentiation
from other locations (Figure 2A). These data suggest that there is less
genetic differentiation between the NF American and the Bear watersheds,
than between the NF and MF American watersheds. We conclude that
measurements of overall genetic differentiation in FYLF from our study
area largely conform to watershed and geographic expectations, with the
exception of the American watershed, which shows a surprisingly high
degree of genetic differentiation between the North (unregulated) and
Middle (hydropeaking) Forks.

To further investigate patterns of genetic variation within the American
Watershed, we performed two PCAs, one on samples from the NF American,
and the other on samples from the MF American. The PCA of the NF
American showed minimal differentiation among locations, with different
study sites blending together and weak patterns of population structure
(Figure 2B). In contrast, PCA of the MF American showed strong
differentiation between sites (Figures 2C, 2D). The MF American PCA
completely resolved all sites, with the first component (PC1) strongly
differentiating the samples in the hydropeaking reach from all other
sites in the MF American. This pattern may be due to the differential
river regulation between the two rivers; the NF American is unregulated
and has weak PCA differentiation, whereas the MF American has a higher
level of river regulation and all sites form distinct genetic clusters,
indicative of reduced gene flow among sites within the MF American.
River regulation is the strongest predictor of genetic isolation with
FYLF in the Northern Sierra

To assess how patterns of genetic differentiation are associated with
river regulation across our entire study area, we estimated pairwise FST
(Wright, 1943) between all collection locations within a river for all
six rivers. We then plotted the scaled mean pairwise FST {[}mean FST /
(1-mean FST){]} (Rousset, 1997) for each location against the mean river
distance (the average distance along the river network from each
collection location to every other location within that study river).
Furthermore, each location was categorized by regulation level of
closest mainstem location (see Methods). While there was a clear
relationship between FST and river distance (as shown by the slope of
regression lines in Figure 3A), there was a striking pattern of elevated
FST by regulation type (Figure 3A). Even the bypass regulation type
showed a distinct pattern of elevated FST. For instance, regulated
rivers with locations separated by less than 10km had FST values
comparable to unregulated locations separated by mean river distances
over 30 km. Hydropeaking was the most extreme pattern of the three
regulation types and showed highly elevated FST values with the steepest
regression coefficient. The baseline FST or global mean increased by
over 0.1 between the unregulated (mean FST=0.141), and regulated
locations (global mean for bypass FST=0.256, hydropeaking FST=0.278).
This suggests a greater degree of isolation within sites in regulated
river reaches compared with FYLF populations in unregulated reaches as
larger FST values represent reductions in heterozygosity due to
population subdivision (Slatkin, 1987). We conclude FYLF in regulated
rivers show patterns of greater population isolation and loss of
heterozygosity compared to frogs in unregulated locations.

To investigate the relative influence of river regulation compared to
other covariates such as river distance on genetic differentiation
(i.e.~FST), we used boosted regression tree (BRT) modeling. Covariates
included flow regime alteration type, river distance, watershed
variables derived from National hydrology data (NHD), topographic data,
and allele frequency spectrum skew (see below, Methods). We found flow
regulation explained the greatest amount of variance in FST (Figure 3B).
Thus, river regulation has a larger relative influence than mean river
distance between sampling locations, which is often the most important
factor influencing genetic differentiation (Rousset, 1997; Slatkin,
1987; Wright, 1943). We conclude there is a pattern of isolation and
limited connectivity between populations in regulated reaches.

\textbf{\emph{River regulation strongly correlated with decreasing
genetic diversity in FYLF}} To investigate the association between river
regulation and genetic diversity trajectory (stable, increasing, or
decreasing), we summarized patterns of genetic variation using two
estimators of θ (4Nμ): Tajima's θ (θ𝜋) is based on the average number of
pairwise differences (Tajima, 1983), and Watterson's θ (θS) is based on
the number of segregating sites (Watterson, 1975) (see methods). These
estimators are influenced by the demographic history of a population and
provide information on the trajectory of changes in genetic diversity.
When genetic diversity has been stable, these estimates should be equal;
when genetic diversity has been increasing, θ𝜋 \textgreater{} θS; and
when genetic diversity has been decreasing, θS \textgreater{} θ𝜋. We
found zero populations sampled within regulated watersheds had evidence
of increasing genetic diversity (e.g., a θ𝜋 − θS that was less than
zero) (Figure 4A). The regulated locations showed a clear trajectory of
genetic diversity loss (Figure 4A, 4B). Three of the four hydropeaking
locations had the highest values of Δθ (θ𝜋 − θS), and the global mean
was significantly different from other regulation types. Although some
tributary populations within unregulated watersheds also showed signs of
genetic diversity loss, the mean genetic diversity trajectory at
unregulated locations was largely neutral (Figure 4B). This indicates
populations in the northern Sierra Nevada which are already limited in
number are losing genetic variation, and river regulation appears to be
exacerbating these patterns. We conclude there is evidence of recent
genetic diversity loss across populations in the regulated river
systems, regardless of regulation type.

\hypertarget{discussion}{%
\section{Discussion}\label{discussion}}

Although massive parallel sequencing (MPS) technologies have the
potential to facilitate collection of high-quality genetic data in
virtually any species, a number of challenges still remain for many
species including low quality or non-existent reference genomes,
large/complex/repetitive genomes, and high cost of processing/sequencing
in studies with many samples. Amphibians are particularly challenging as
many species have very large genome sizes (Nunziata, Lance, Scott,
Lemmon, \& Weisrock, 2017) for example, there are only two frog
reference genome assemblies available as of 2018 (Hellsten et al., 2010;
Sun et al., 2015). Our results demonstrate that Rapture (Ali et al.,
2016) is a suitable method to rapidly and inexpensively discover a large
number of loci in a frog species with a complex genome. In this study,
we used new RAD sequencing and RAD capture (Rapture) methods (Ali et
al., 2016) to generate high-quality genomic data suitable for
discovering and genotyping many single nucleotide polymorphisms (SNPs)
in FYLF. Based on this dataset, we were able to successfully
characterize patterns of genetic variation within FYLF as well as design
a set of RAD capture baits that can be used as a genetic monitoring
resource for FYLF (and likely other ranid species). This highlights that
the collection of genetic information, even from large numbers of
samples or in complex genomes, is no longer a limitation with current
genomic methods such as RAD and Rapture.

Demographic connectivity is widely recognized as a fundamental driver of
long-term population persistence (Fahrig \& Merriam, 1985; Taylor,
Fahrig, Henein, \& Merriam, 1993). Populations must adapt over time and
connectivity is a major way to transfer genetic information. For
example, previous studies have shown that adaptation can occur by
spreading specific alleles across large geographic distances (Miller et
al., 2012; Prince et al., 2017). In many regulated river reaches in the
Sierra Nevada, FYLF now occur in isolated locations, breeding in
tributaries rather than mainstem habitats. However, since these frogs
have the potential to move long distances (FYLF have been observed
moving over 1 km per day (Bourque, 2008)), the extent to which current
population connectivity has been lost due to river regulation remains
unknown. Examining pairwise FST, revealed a major decrease in
connectivity in populations in regulated systems, even with limited
river regulation (i.e., bypass reaches). Usually isolation by distance
patterns best describe variation in genetic data, yet the primary factor
influencing genetic differentiation among these rivers is hydrologic
alteration (Figure 3B). Thus, despite being able to move long distances,
FYLF have not been able to maintain population connectivity in regulated
rivers. This demonstrates that even in species that can move relatively
long distances and pass potential physical barriers (e.g.,
infrastructure such as dams, canals, and reservoirs likely do not
represent barriers to movement of FYLF) loss of connectivity may still
occur and can be revealed with genetic analysis.

Genetic diversity is also a critical component for long-term population
persistence because it is closely related to the evolutionary capacity
for adaptation to environmental changes (Frankham, 2002; Hoffmann \&
Sgrò, 2011; Ishiyama, Koizumi, Yuta, \& Nakamura, 2015; Lande \&
Shannon, 1996). In some cases, isolated populations can maintain genetic
diversity when they are sufficiently sized (Whiteley et al., 2010),
however, species of conservation concern typically have small and/or
declining populations and thus may be susceptible to genetic diversity
loss (Frankham, 2002; Krohn et al., 2018). Throughout the Sierra Nevada,
FYLF have largely disappeared from regulated mainstem rivers, but the
extent to which existing populations have been able to maintain genetic
diversity is unclear. Strikingly, our analysis revealed that every
single population within the regulated watersheds exhibits a trajectory
of genetic diversity loss. Thus, genomic analysis of molecular variation
provides a powerful lens to discover and assess trajectories of genetic
diversity.

Our analyses, using metrics that serve as a reasonable proxy for genetic
health, does not bode well for the long-term persistence of FYLF
populations in regulated rivers in the Sierra Nevada. Many of these FYLF
populations are already losing genetic diversity and given their small
size and reduced connectivity the effects of inbreeding will likely
exacerbate their problems. FYLF have evolved in river systems with
consistent hydrologic seasonality and predictability, despite
inter-annual variation. Flow regulation has altered patterns of
hydrologic seasonality and predictability in many watersheds (Sarah J.
Kupferberg et al., 2012). Long-term population persistence may still be
possible if conservation efforts utilize methods that promote or
maintain genetic health and increase population connectivity. For
example, simulations by Botero et al. (2015) demonstrated adaptation
persisted in modeled populations through large environmental
changes---if phenotypic strategies were appropriate before and after the
change---but modeled populations declined rapidly when the current
strategy was a mismatch to the current environment. Thus, FYLF
conservation efforts should focus on river reaches where flow management
may provide opportunities to more closely mimic local natural flow
regimes and thus improve hydrologic connectivity.

Detecting evolutionary responses to within- and among-year changes in an
ecological or hydrological context has previously been difficult.
However, utilizing genetic data can fill these gaps and provide a highly
informative process for identifying the impacts of anthropogenic and
environmental change on the process of adaptation (Botero et al., 2015;
Kahilainen, Puurtinen, \& Kotiaho, 2014). We demonstrate that an aquatic
species that has adapted to local hydrology patterns shows poor genetic
health (i.e., clear patterns of decreased connectivity and trajectories
of genetic diversity loss). Our results highlight the potential impact
of river regulation on aquatic organisms and their potential for long
term persistence. In the future, similar genetic approaches could be
used in many other contexts to explore the impacts of river regulation
on aquatic organisms. Taken together, our results demonstrate that
genetic monitoring can be a powerful tool for assessment of population
health and should be a critical component of conservation management in
aquatic organisms.

\textbf{Data Archiving Statement:} Should the manuscript be accepted,
the data supporting the results will be archived in an appropriate
public repository such as Dryad, and the data DOI will be appended to
the end of the article.

\texttt{\{r\ maxdelays,\ results="asis",\ echo=Fevalrary(knitr)\ kable(max\_delays,\ \ \ \ \ \ \ \ col.names\ =\ c("Airline",\ "Max\ Arrival\ Delay"),\ \ \ \ \ \ \ caption\ =\ "Maximum\ Delays\ by\ Airline",\ \ \ \ \ \ \ caption.short\ =\ "Max\ Delays\ by\ Airline",\ \ \ \ \ \ \ longtable\ =\ TRUE,\ \ \ \ \ \ \ booktabs\ =\ TRUE)}

The last two options make the table a little easier-to-read.

\hypertarget{hybrids}{%
\chapter{Hybridization between two sympatric ranid frog species in the
northern Sierra Nevada, California}\label{hybrids}}

\hypertarget{introduction-1}{%
\section{Introduction}\label{introduction-1}}

Landscape changes can influence species demography and migration
patterns (Li et al. 2017) which can change rates of gene flow within
species. Changing migration rates and population sizes can influence
population structure; thus, over time, landscape changes can cause
significant changes in genetic diversity within a species. Furthermore,
cross-breeding or hybridization between closely related taxa can promote
gene flow (introgression) between species, which may be an important
evolutionary mechanism for either homogenization (reversing initial
divergence between species), speciation (from reproductive isolation of
hybrid populations), or adaptation (transfer of adaptive alleles)
(Mallet 2007, Abbott et al. 2013, Barrera-Guzmán et al. 2018).

Hybridization events in vertebrates may be rare, or rarely detected, and
thus identifying potential hybridization can be difficult and may be
affected by sampling design, timing, and resolution of genetic markers.
Therefore, occurrences of hybridization likely remain unknown,
particularly in cryptic taxa. Assessing population admixture or
detecting potential hybridization has previously been challenging;
however, modern genetic methods provide a powerful approach to assess
populations at fine geographic and evolutionary scales (Ali et al. 2016,
Prince et al. 2017).

We investigate the potential for hybridization in two sympatrically
occurring endemic frog species in the Sierra Nevada of California.
Foothill yellow-legged frogs, \emph{Rana boylii}, (Baird 1856)
historically occurred in lower and mid-elevation (\textless{}1500 m)
streams and rivers from Southern Oregon to northern Baja California west
of the Sierra-Cascade crest (Stebbins 2003), whereas Sierra Nevada
yellow-legged frogs, \emph{Rana sierrae}, (Camp 1917) typically occurred
from 1500 m to over 3600 m in lakes and streams (Stebbins 2003,
@zweifel\_ecology\_1955). Population declines have been documented
across the former range of both of these species; \emph{R. sierrae} has
been extirpated from over 90 percent of its historical range (Drost and
Fellers 1996) (Vredenburg 2004) while \emph{R. boylii} has been
extirpated from 50 percent of its historical range (Jennings and Hayes
1994, Davidson et al.~2002). Both species are of conservation concern;
in 2014, the U. S. Fish and Wildlife Service (USFWS) listed \emph{R.
sierrae} as endangered under the U. S. Endangered Species Act (ESA)
(USFWS 2014), and \emph{R. boylii} is listed as a species of special
concern in California and is a candidate for listing under the
California and federal ESAs.

Unlike other ranid frog species with broad areas of potential
intergradation (Shaffer et al. 2004), \emph{R. boylii} and \emph{R.
sierrae} only rarely occur sympatrically. Zweifel (1955) described one
historical location where these two species co-occurred, in Butte County
near DeSabla. Currently the only known location where both species are
found is several tributaries to the Feather River in the northern Sierra
Nevada, California (Figure 1). Hybridization between these species has
not previously been documented. Furthermore, breeding experiments by
Zweifel (1955) between \emph{R. sierrae} (formerly known as \emph{R.
muscosa}) and \emph{R. boylii} yielded very low viability in
fertilization and high incidences of embryological
abnormalities---indicating a post-zygotic barrier between the species.
However, these experiments only crossed female \emph{R. sierrae} with
male \emph{R. boylii}, and the individuals were from very different
California regions (e.g., Butte and Nevada County vs.~Contra Costa
County). \emph{Rana boylii} and \emph{R. sierrae} species have very
similar morphology and habitat preferences in areas where they co-occur;
thus assigning individuals to species is difficult and imprecise using
field identification methods. This presents a challenge for management
because these sympatric species have different conservation status and
management objectives. We employed modern genetic methodology to better
understand \emph{R. sierrae} and \emph{R. boylii} where their ranges
overlap. We investigated three primary questions:
\begin{enumerate}
\def\labelenumi{\arabic{enumi}.}
\tightlist
\item
  Can hybridization be detected between two sympatrically occurring
  threatened and endangered (ESA) frog species in the Sierra Nevada
  using data generated from genome-wide single nucleotide polymorphisms
  (SNPs);
\item
  If hybrids can be detected, do genetic signatures suggest hybrid
  viability (i.e., can hybrids reproduce, leading to introgression
  between species);
\item
  Using coalescent modeling, are migration rates between species in
  sympatrically occurring populations higher than in allopatrically
  occurring populations in adjacent watersheds?
\end{enumerate}
\hypertarget{materials-and-methods}{%
\section{Materials and Methods}\label{materials-and-methods}}

\hypertarget{sampling-and-dna-extraction}{%
\subsection{Sampling and DNA
Extraction}\label{sampling-and-dna-extraction}}

To investigate potential hybridization between \emph{R. sierrae} and
\emph{R. boylii}, a total of 458 tadpole tail clips, buccal swabs, or
tissue samples were compiled. Samples were identified to species in the
field as either \emph{R. boylii}, \emph{R. sierrae}, or ``unknown'',
which were individuals which could not be visually confirmed as either
species (Stebbins 2003). The samples were collected between 1992 and
2016, from three watersheds in the Sierra Nevada (the Feather, Yuba, and
American) (Table S1). All unknown individuals were from Feather
watershed localities.
\begin{landscape}
\begin{longtable}[t]{l|l|r|l|l|l|l}
\caption{\label{tab:tableS1}Sample Sites}\\
\hline
Locality & River & No. Samples & Lat. & Lon. & Elev (m) & Watershed (HUC8)\\
\hline
\endfirsthead
\caption[]{\label{tab:tableS1}Sample Sites \textit{(continued)}}\\
\hline
Locality & River & No. Samples & Lat. & Lon. & Elev (m) & Watershed (HUC8)\\
\hline
\endhead
MFA-AMEC & MFA & 5 & 38.934 & -120.9436 & 240 & American\\
\hline
MFA-GASC & MFA & 1 & 38.9665 & -120.9325 & 242 & American\\
\hline
MFA-TODC & MFA & 6 & 38.9638 & -120.9216 & 368 & American\\
\hline
MFA-US-R & MFA & 1 & 39.0075 & -120.7316 & 360 & American\\
\hline
NFA & NFA & 12 & 39.1079 & -120.9227 & 363 & American\\
\hline
NFA-BUNC & NFA & 8 & 39.0376 & -120.9103 & 286 & American\\
\hline
NFA-EUCHDS & NFA & 3 & 39.1849 & -120.762 & 580 & American\\
\hline
NFA-INDC & NFA & 3 & 39.0567 & -120.9085 & 296 & American\\
\hline
NFA-LyonsPk & NFA & 11 & 39.2067 & -120.3113 & 2529 & American\\
\hline
NFA-POND & NFA & 2 & 38.9999 & -120.9406 & 241 & American\\
\hline
NFA-ROBR & NFA & 6 & 39.1045 & -120.9267 & 400 & American\\
\hline
NFA-SHIC & NFA & 8 & 39.0417 & -120.9009 & 341 & American\\
\hline
NFA-SLAR & NFA & 5 & 39.0987 & -120.9255 & 356 & American\\
\hline
NFMFA-SC & NFMFA & 7 & 39.0224 & -120.7369 & 522 & American\\
\hline
RUB-HighlandDS & RUB & 18 & 38.9615 & -120.2422 & 2312 & American\\
\hline
RUB-HighlandLk & RUB & 29 & 38.9573 & -120.2418 & 2383 & American\\
\hline
RUB-LC-US & RUB & 1 & 38.9889 & -120.69 & 415 & American\\
\hline
RUB-USPH & RUB & 5 & 38.9993 & -120.7233 & 361 & American\\
\hline
RUB-Zitella & RUB & 9 & 38.9595 & -120.227 & 2335 & American\\
\hline
RUB-ZitellaLk & RUB & 2 & 38.9604 & -120.2316 & 2337 & American\\
\hline
SFA-CAMI & SFA & 2 & 38.8115 & -120.5787 & 725 & American\\
\hline
FEA-BeanCk & FEA & 60 & 39.9774 & -121.091 & 1397 & Feather\\
\hline
FEA-EBNFF & FEA & 6 & Unknown & Unknown & Unknown & Feather\\
\hline
FEA-GoldLk & FEA & 4 & 39.9416 & -121.136 & 1816 & Feather\\
\hline
FEA-LoneRockCk & FEA & 3 & 40.2012 & -120.6453 & 1563 & Feather\\
\hline
FEA-MillCk & FEA & 4 & 39.9591 & -121.1573 & 1891 & Feather\\
\hline
FEA-RockLkBucksCk & FEA & 1 & 39.9403 & -121.1499 & 2102 & Feather\\
\hline
FEA-RockLkSilver & FEA & 8 & 39.9409 & -121.1422 & 1902 & Feather\\
\hline
FEA-SFRockCk & FEA & 27 & 39.8789 & -121.0022 & 1470 & Feather\\
\hline
FEA-SPANISH-BGulch & FEA & 5 & 39.9546 & -121.089 & 1283 & Feather\\
\hline
FEA-SPANISH-RockCk & FEA & 1 & 39.9445 & -121.0221 & 1090 & Feather\\
\hline
FEA-SPANISH-SilverCk & FEA & 3 & 39.9377 & -121.0849 & 1189 & Feather\\
\hline
FEA-SPANISH-Wapaunsie & FEA & 1 & 39.9523 & -121.0373 & 1096 & Feather\\
\hline
FEA-SpanishCk & FEA & 26 & 39.9541 & -121.0541 & 1192 & Feather\\
\hline
NFF-Poe & NFF & 1 & 39.736 & -121.4702 & 284 & Feather\\
\hline
FORD-Mossy-P1 & FORD & 1 & 39.381 & -120.4623 & 2157 & Yuba\\
\hline
FORD-Mossy-P2 & FORD & 3 & 39.3852 & -120.4714 & 1998 & Yuba\\
\hline
FORD-Mossy-P3 & FORD & 2 & 39.3765 & -120.4603 & 2158 & Yuba\\
\hline
FORD-MossyDS & FORD & 19 & 39.3853 & -120.4728 & 1984 & Yuba\\
\hline
FORD-MossyPond & FORD & 34 & 39.3781 & -120.4701 & 2106 & Yuba\\
\hline
FORD-NorthCkTrib & FORD & 34 & 39.3869 & -120.451 & 2090 & Yuba\\
\hline
MFY-OREGCk & MFY & 10 & 39.4419 & -121.0575 & 620 & Yuba\\
\hline
MFY-Remmington & MFY & 1 & 39.4137 & -120.9912 & 620 & Yuba\\
\hline
MFY-US-OH & MFY & 7 & 39.413 & -120.9903 & 624 & Yuba\\
\hline
NFY & NFY & 12 & 39.5119 & -120.9774 & 705 & Yuba\\
\hline
NFY-SLATE-CGRav & NFY & 3 & 39.6928 & -120.9258 & 1457 & Yuba\\
\hline
NFY-SLATE-Onion & NFY & 3 & 39.6355 & -121.0395 & 1300 & Yuba\\
\hline
SFY & SFY & 1 & 39.3539 & -120.7342 & 890 & Yuba\\
\hline
SFY-FallCk & SFY & 5 & 39.3553 & -120.7371 & 884 & Yuba\\
\hline
SFY-HUMBUG & SFY & 1 & 39.3637 & -120.921 & 877 & Yuba\\
\hline
SFY-LOGA & SFY & 3 & 39.3691 & -120.8526 & 1201 & Yuba\\
\hline
SFY-MCKI & SFY & 3 & 39.368 & -120.8354 & 1084 & Yuba\\
\hline
SFY-MISC & SFY & 7 & 39.361 & -120.8814 & 1095 & Yuba\\
\hline
SFY-RockCk & SFY & 3 & 39.3298 & -120.9863 & 594 & Yuba\\
\hline
SFY-Scotchman & SFY & 3 & 39.3293 & -120.777 & 1167 & Yuba\\
\hline
SFY-ShadyCk & SFY & 9 & 39.3543 & -121.059 & 675 & Yuba\\
\hline
\end{longtable}
\end{landscape}
The Yuba and American watersheds share a similar Mediterranean climate,
underlying geology, watershed aspect (west-slope), and vegetative
communities. The Feather watershed shares a similar climate but has a
slightly different underlying geology and aspect than that of other
watersheds in the Sierra Nevada. The Feather watershed lies in the
transition zone of the northern Sierra Nevada and the Cascades/Basin and
Range Province, and thus the landscape in the northern portion of the
watershed is comprised largely of volcanic bedrock while the southern
portion is largely granitic (Durrell 1988). Field sampling was conducted
following methods in Heyer et al. (1994) under CDFW SCP Permit \#0006881
and Federal permit TE-40087B-0 with IACUC protocol \#19327 and
\#04718-001. Individual post-metamorphic frogs were buccal-swabbed
following established protocols (Goldberg et al.~2003, Pidancier et
al.~2003, Broquet et al.~2007). Each post-metamorphic individual was
comprehensively swabbed underneath tongue and inside of both cheeks for
approximately 30 sec to one minute. Swabs were air dried for
approximately five minutes and placed in 1.5 mL microcentrifuge tubes
while in the field or placed in lysis buffer (Goldberg et al.~2003).
Dried samples were stored in the laboratory at -80°C until DNA
extraction. Where possible, tail clips from tadpole larvae were
collected, and tadpoles greater than 15 mm total length were
targeted(Wilbur and Semlitsch 1990, Parris et al.~2010). One clip was
taken per individual tadpole and dried on Whatman filter paper (grade 1)
and stored at room temperature or in 95\% ethanol. DNA was extracted
from ethanol-stored samples using Qiagen DNeasy kits following
manufacturer protocol and stored at -20°C. DNA was extracted from dried
buccal swabs and tail clips using an Ampure magnetic bead-based protocol
(Ali et al.~2016) and stored at -20°C.

\hypertarget{rapture-sequencing-1}{%
\subsection{Rapture Sequencing}\label{rapture-sequencing-1}}

To produce a high-quality genomic resource for frog species with large
genome sizes, we interrogated a significant fraction of the \emph{R.
boylii} genome using RAD sequencing with SbfI (Miller et al.~2007, Baird
et al.~2008, Ali et al.~2016). Paired-end sequence data were generated
using 24 \emph{R. boylii} individuals (Table S2). RAD libraries were
constructed following the protocol described in (Ali et al.~2016). De
novo locus discovery and contig extension were carried out as previously
described (Miller et al.~2012) using the alignment program Novoalign and
the genome assembler PRICE (Ruby et al.~2013). This resulted in a set of
77,544 RAD contigs ranging from 300 to 800 bp which served as a de novo
partial genome reference for all subsequent downstream analyses (File
S3: de novo). We next removed loci with five or more SNPs, and randomly
selected 10,000 loci from the remaining subset. Of these 10,000 loci,
8,533 were successfully designed into 120 bp RAD capture baits by Arbor
Biosciences (File S4: Baits). Sample libraries were prepared for
sequencing following RAD Capture (Rapture) methods outlined in Ali et
al. (2016). These samples were then then used to identify putative
high-quality SNPs following sequencing.

Sampled individuals were aligned against the de novo partial genome
reference using the BWA-MEM algorithm (Li and Durbin 2010, Li 2013), and
converted to BAM format and filtered for properly paired alignments
using Samtools (File S1). Next, alignments from three different
sequencing runs on an Illumina HiSeq were merged together and duplicates
were removed using Samtools (Li et al.~2009). For all downstream
analysis, we selected individuals that had greater than 25,000
alignments (n=311), which provided sufficient data to investigate
population genetic attributes at broad and fine geographic scales (Table
S5). To generate SNP (i.e., segregating site) data, a probabilistic
framework was used for all population genetic analyses as it does not
require calling genotypes and is suitable for low-coverage sequencing
data (Korneliussen et al.~2013, Fumagalli et al.~2013). SNP discovery,
minor allele frequencies (MAF) estimates, and genotype probabilities
were conducted using ANGSD (Korneliussen et al.~2014). ANGSD analyses
were conducted following methods from Prince et al. (2017), with a
minimum mapping quality score (minMapQ) of 10, a minimum base quality
score (minQ) of 20, the genotype likelihood model (GL 1), specifying the
RAPTURE bait locations using the -sites flag, and only sites represented
in at least 50\% of the included samples (minInd) were used.
Furthermore, genomic sites were designated as polymorphic only if MAFs
were greater than 0.05 and the probability of the site not being
polymorphic was less than 1e-6. Using this approach, over 44,000
polymorphic sites were identified across all \emph{R. boylii} study
samples.

\hypertarget{pca-admixture}{%
\subsection{PCA \& Admixture}\label{pca-admixture}}

To assess population structure and coancestry, ANGSD was used to
generate PCA and NGSadmix was used to calculate admixture. Settings used
in ANGSD for PCA to identify polymorphic sites included a SNP\_pval of
1e-6, inferring major and minor alleles (doMajorMinor 1), estimating
genotypic likelihoods (GL 1), estimating allele frequencies (doMaf 2)
(Kim et al.~2011), retaining SNPs with a minor allele frequency of at
least 0.05 (minMaf), specifying the RAPTURE bait locations using the
-sites flag, estimation of genotype posterior probabilities using a
uniform prior (doPost 2), and the -doIBS 1 and -doCov 1 options.
Principal components (PC) summarizing population structure were derived
from classic eigenvalue decomposition and were visualized using the
ggplot2 package in R (R Core Team 2017). To assess admixture between
\emph{R. sierrae} and \emph{R. boylii}, genotype likelihood data (-GL 2)
was generated in ANGSD with the same settings as above, in addition to
retaining only SNPs that were shared in at least of 50\% of the samples,
-doPost 2, -doGLF 2, and limiting to higher quality alignment data
(-minMapQ 10, -minQ 20). We then used NGSadmix (Skotte et al.~2013) to
infer ancestry proportions in \emph{R. sierrae} and \emph{R. boylii}
individuals. NGSadmix is a robust admixture method that can be applied
to low-depth NGS data, and does not require called genotypes, thus
reducing error associated with potential ascertainment and uncertainty
in the data (Skotte et al.~2013).

\hypertarget{f1-vs-f2-test-with-species-diagnostic-snps}{%
\subsection{F1 vs F2 Test with Species Diagnostic
SNPs}\label{f1-vs-f2-test-with-species-diagnostic-snps}}

To test whether hybrids were first generation filial (F1) hybrids or
progeny from F1 hybrids from subsequent generations (e.g., F2, F3,
etc.), we identified differentially fixed (i.e., species-specific) SNPs
and assessed heterozygosity at these loci in hybrid individuals as F1
vs.~F2 hybrid individuals will have different degrees of heterozygosity
in these species-diagnostic SNPs. We called genotypes in ANGSD using a
uniform prior (-doPost 2) and the following settings: -GL 1, -doGeno 13,
-postCutoff 0.95, -doMaf 1, -doMajorMinor 1, -minInd 2, -SNP\_pval 1e-6,
-minMapQ 20, -minQ 20, and specifying the RAPTURE bait locations using
the -sites. The subsequent output (*.geno.gz) was then processed in the
program R using the dplyr package (Wickham et al.~2018) to manipulate
and filter to homozygous diagnostic SNPs. Data were filtered to include
only loci where over 50 non-hybrid individuals from each species had
called genotypes at a given polymorphism.

\hypertarget{demographic-modeling-with-fastsimcoal2}{%
\subsection{Demographic Modeling with
fastsimcoal2}\label{demographic-modeling-with-fastsimcoal2}}

To quantify divergence times and migration rates between \emph{R.
sierrae} and \emph{R. boylii}, we used coalescent simulations in
fastsimcoal2 (Excoffier and Foll 2011, Excoffier et al.~2013). This
maximum-likelihood modeling approach uses simulations to estimate the
expected site-frequency spectra (SFS) for a demographic model of
interest to calculate a composite likelihood, and then utilizes a
maximization procedure to find the maximum-likelihood parameter
estimates.

We calculated folded joint SFS for each species in each watershed from
SNP data generated from ANGSD because the ancestral condition is
unknown. For all models, we assumed the potential for bidirectional gene
flow and that extant genetic clusters emerged simultaneously from a
common ancestry. We tested models that allowed for population growth,
and models with no growth. We used two conservative model scenarios to
estimate divergence times and migration rates between species in each
watershed. To estimate migration probabilities per generation between
species within each watershed, we set the divergence time parameters
between 1--1.1 billion years ago to create simplified migration-only
models. To estimate divergence time between species, we used the
watershed that had the lowest migration rate from the previous
migration-only models, and generated divergence time estimates assuming
no migration between species.

The basic steps taken to obtain final model estimates from fastsimcoal2
used a set of 10 replicate models, followed by comparison of maximum
observed and expected likelihoods to select the best-fit model (Akaike
1973), then simulate new SFS using the best-fit model for parametric
bootstrapping. Following Excoffier and Foll (2011), we used 1,000
randomly drawn SNPs from each SFS to generate 100,000 coalescent
simulations for likelihood calculations (estimation of the expected SFS)
with a maximum of 40 cycles for the conditional maximization algorithm.
To select the best-fit model we selected the model replicate that
minimized the difference between the maximum expected likelihood and the
maximum observed likelihood. We used parametric bootstrapping to
generate 95\% confidence intervals for each best-fit model using 100
bootstraps for each model and selecting the best model from each
bootstrap based on maximum likelihoods as described above.

\hypertarget{results-1}{%
\section{Results}\label{results-1}}

\hypertarget{rapture-produced-high-quality-genomic-data-for-both-r.-sierrae-and-r.-boylii}{%
\subsection{\texorpdfstring{Rapture produced high quality genomic data
for both \emph{R. sierrae} and \emph{R.
boylii}}{Rapture produced high quality genomic data for both R. sierrae and R. boylii}}\label{rapture-produced-high-quality-genomic-data-for-both-r.-sierrae-and-r.-boylii}}

Individual samples were collected across 56 different sampling
localities in three different watersheds (Figure 1, Table S1). For
downstream analysis, we filtered and retained 311 samples from the
original sequencing data that contained a minimum of 25,000 alignments
(Table S5). The final merged dataset mean alignments per sample was
229,485 (Table S5), and the mean number of samples per site was eight.
These frog species are cryptic, and often occur in low densities, so we
retained all sites in our analysis, regardless of the number of samples
(Table S1). We conclude that the sequence data we obtained should be
appropriate for population genetic analyses across our study area.



\begin{figure}
\includegraphics[angle=90, scale=.75]{figure/figure_01_overview_hybrid} \caption{Map of sampling locations in the Feather, Yuba, and
American watersheds. RABO=\emph{R. boylii}, RASI=\emph{R. sierrae}.}\label{fig:fig1map}
\end{figure}
\hypertarget{rangewide}{%
\chapter{\texorpdfstring{Refining conservation unit boundaries of a
sentinel stream-breeding frog (\emph{Rana boylii}) using population
genomics}{Refining conservation unit boundaries of a sentinel stream-breeding frog (Rana boylii) using population genomics}}\label{rangewide}}
\begin{Shaded}
\begin{Highlighting}[]
\CommentTok{#include_graphics(path = "figure/uw.png")}

\CommentTok{# Here is a reference to the UW logo: Figure \textbackslash{}@ref(fig:uwlogo).  Note the use of the `fig:` code here.}
\end{Highlighting}
\end{Shaded}
\hypertarget{introduction-2}{%
\section{INTRODUCTION}\label{introduction-2}}

The use of modern genomic sequencing technology has greatly advanced the
ability for higher resolution analyses of both geographic and ecological
patterns in populations (Nunziata et al. 2017, Barbosa et al. 2018,
Hendricks et al. 2018). Reduced representation sequencing methods such
as restriction site-associated DNA sequencing (RADSeq) (Miller et al.
2007, Baird et al. 2008, Ali et al. 2016) provides a powerful tool to
address ecological genomics questions at scales that were previously
impossible using traditional field methods. Furthermore, new methods
such as RAD Capture (Rapture) (Ali et al. 2016) adapt RADSeq to target
desired loci and allow highly efficient genotyping of thousands of
individuals at once. As historical and future landscape use can
influence species demography and migration patterns (Burkey 1989,
Anderson and Beer 2009, Barbosa et al. 2018), these genomic tools will
be invaluable for assessing critical factors for long-term persistence
in sensitive populations or species.

The ecological integrity of freshwater systems and their constituent
biota are rapidly declining globally (Ricciardi and Rasmussen 1999), and
conservation efforts will require assessment of the adaptive capacity of
populations to rapid environmental change. Given limited capacity to
conserve, it is important to define and establish clear geographic
boundaries for conservation units such as distinct population segments
across a species' range. Delineation of distinct population segments can
be used for prioritizing objectives in conservation management.
Furthermore, quantification and comparison of relative genetic diversity
within and among populations can provide additional information as a
benchmark for future assessment responses to conservation actions. Thus,
quantifying and linking landscape change with genetic diversity metrics
may provide an important baseline to track how sensitive populations
respond to future environmental change (through reduced adaptive
potential) as well as evaluating whether restoration efforts are
effective (i.e., increasing genetic connectivity, diversity, effective
breeder/population size).

Amphibians are particularly sensitive to changes in the ecosystem due to
their physiology and ontogeny (Davidson et al. 2002, Beebee and
Griffiths 2005), thus the ability to utilize environmental variables as
life history cues can be especially important. In highly dynamic
riverine environments, organisms must constantly adapt to temporal and
spatial changes. One such sentinel stream-breeding species is the
Foothill yellow-legged frog (\emph{Rana boylii}), a native to California
and Oregon which historically occurred in lower elevation (0-1500m)
streams and rivers from Southern Oregon to northern Baja California west
of the Sierra-Cascade crest (Stebbins 2003). As a lotic breeding
amphibian, \emph{R. boylii} is tied closely to the local hydrology in
watersheds it inhabits, and therefore it is particularly sensitive to
alterations to flow regimes (Kupferberg 1996, Lind et al. 1996,
Kupferberg et al. 2012).

As with many amphibians in California (Davidson 2004, Peek 2010, Thomson
et al. 2016), there have been significant population declines across the
former range of this species, particularly in southern California and
the Sierra Nevada where it has been extirpated from approximately 50
percent of its historical range (Jennings and Hayes 1994, Davidson et
al. 2002). \emph{Rana boylii}, currently designated as a species of
special concern (CDFW) in the state of CA, has been petitioned as
candidate for listing under the federal (USFWS) Endangered Species Act
(USFWS 2014) as well as the state (CDFW) Endangered Species Act.

Effective conservation management of this species will need to consider
and prioritize maintenance of genetic diversity as part of any listing
decision because it is closely related to the evolutionary capacity for
adaptation to environmental changes (Lande and Shannon 1996). Thus,
utilizing genetic data provides a potentially informative process for
identifying the impacts of anthropogenic and environmental change on the
process of adaptation. Establishing high-resolution genetic boundaries
for populations across the species range as well as quantification of
relative genomic diversity metrics (i.e., genomic diversity, population
connectivity) would help managers prioritize conservation actions. A
recent study by McCartney-Melstad et al. (2018) identified five major
clades in \emph{R. boylii} with strong geographically structured genetic
subdivision across its range in California and Oregon. Here we provide
an additional population genomic analysis across the range of this
declining sentinel stream species that is currently a candidate for
listing. We provide additional geographic and genetic resolution to
McCartney-Melstad et al. (2018), as well as quantify genetic diversity
metrics across subpopulations and clades as both a reference and
assessment of the potential for long-term persistence across this
species' range.

\hypertarget{materials-and-methods-1}{%
\section{MATERIALS AND METHODS}\label{materials-and-methods-1}}

\hypertarget{sampling-and-dna-extraction-1}{%
\subsection{Sampling and DNA
extraction}\label{sampling-and-dna-extraction-1}}

A total of 1103 individual tadpole tail clips, buccal swabs, or tissue
samples were compiled, collected between 1992 and 2016 across the range
of \emph{R. boylii}. Field sampling was conducted following methods in
Heyer et al. (1994) under CDFW SCP Permit \#0006881, with IACUC protocol
\#19327. Individual post-metamorphic frogs were buccal-swabbed following
established protocols (Goldberg et al. 2003, Pidancier et al. 2003,
Broquet et al. 2007). Each post-metamorphic individual was
comprehensively swabbed underneath tongue and cheek for approximately
one minute. Swabs were air dried for approximately five minutes and
placed in 1.5 mL microcentrifuge tubes while in the field. Samples were
stored in the laboratory at -80°C until DNA extraction. Where possible,
tail clips from tadpole larvae were collected, and tadpoles greater than
15 mm total length were targeted (Wilbur and Semlitsch 1990, Parris et
al. 2010). One clip was taken per individual tadpole and dried on
Whatman filter paper (grade 1) and stored at room temperature. Some
older tissue samples consisted of toe clips placed in 100\% ethanol for
storage, and DNA extraction from these samples used Qiagen DNeasy kits
following the manufacturer's protocol. Buccal swabs and tail clip DNA
were extracted using an Ampure magnetic bead-based protocol (Ali et al.
2016). DNA samples were stored at -20°C.

\hypertarget{generating-high-quality-sequencing-data}{%
\subsection{Generating high-quality sequencing
data}\label{generating-high-quality-sequencing-data}}

To produce a high-quality genomic resource for frog species with large
genome sizes, we interrogated a significant fraction of the \emph{R.
boylii} genome using a SbfI restriction enzyme and high-density RAD
sequencing on an Illumina HiSeq (Miller et al. 2007, Baird et al. 2008).
Paired-end sequence data were generated using 24 \emph{R. boylii}
individuals (\textbf{Table S1}). RAD libraries were constructed
following the protocol described in Ali et al. (2016). De novo loci
discovery and contig extension were carried out via custom PERL scripts
(Miller et al. 2012), the alignment program Novoalign and the genome
assembler PRICE (Ruby et al. 2013). This pipeline resulted in a set of
77,544 RAD contigs ranging from 300 to 800 bp which served as a de novo
partial genome reference for all subsequent downstream analyses
(\textbf{Supplemental File S2}). Using these data, we filtered data to
loci with 4 or fewer SNPs, and randomly selected 10,000 loci from this
subset. Using these RADSeq data, 8,533 RAD capture baits (120bp) were
designed by Arbor Biosciences from the de novo alignment
(\textbf{Supplemental File S3}). The number of polymorphic loci
identified across all \emph{R. boylii} study samples was 44,406. RAPTURE
was then used to identify putative high-quality SNPs.

Three different sequencing runs on an Illumina HiSeq were merged
together, filtered, and duplicates were removed using ANGSD and Samtools
(Li et al. 2009). Sampled individuals were aligned against the de novo
partial genome reference using the BWA-MEM algorithm (Li and Durbin
2010, Li 2013) and saved to BAM format. To generate SNP (segregating
site) data, a probabilistic framework was used for all population
genetic analyses as it does not require calling genotypes and is
suitable for low-coverage sequencing data (Korneliussen et al. 2013,
Fumagalli et al. 2013). Estimates of per site minor allele frequencies
(MAF), genotype probabilities and SNP discovery were conducted using
ANGSD and NGStools (Korneliussen et al. 2014, Fumagalli et al. 2014).
Genomic sites were designated as polymorphic only if MAFs were greater
than 0.05 and the probability of the site not being polymorphic was less
than 10-12. ANGSD analyses were conducted following methods from Prince
et al. (2017), with a minimum mapping quality score (minMapQ) of 10, a
minimum base quality score (minQ) of 20, the genotype likelihood model
(GL 1), and only sites represented in at least 50\% of the included
samples (minInd) were used (Li and Durbin 2011).

\hypertarget{quantifying-genetic-structure}{%
\subsection{Quantifying genetic
structure}\label{quantifying-genetic-structure}}

To characterize and quantify genetic population structure within and
among watersheds, we conducted principal component analysis (PCA) using
data subsampled to different alignment thresholds (e.g., all individuals
with a minimum of 100,000 alignments) to determine the amount of data
needed for population analyses. For downstream analysis, we selected
individuals that had greater than 100,000 alignments. To assess
population structure and coancestry, ANGSD was used to generate PCA and
NGSadmix was used to estimate admixture. Settings used in ANGSD for PCA
to identify polymorphic sites included a SNP\_pval of 1e-6, inferring
major and minor alleles (doMajorMinor 1), estimating allele frequencies
(doMaf 2) (Kim et al. 2011), retaining SNPs with a minor allele
frequency of at least 0.05 (minMaf), estimation of genotype posterior
probabilities using a uniform prior (doPost 2), specifying the RAPTURE
bait locations using the -sites flag, calculating the PCA matrix with
the -doIBS 1 and -doCov 1 options, and limiting the analysis to higher
quality alignment data (-minMapQ 10, -minQ 20). Principal components
(PC) summarizing population structure were derived from classic
eigenvalue decomposition and visualized using the ggplot2 package in R
(R Core Team 2017). To assess admixture in \emph{R. boylii}, genotype
likelihood data (-GL 2 and -doGLF 2) was generated in ANGSD with the
same settings as above. We then used NGSadmix (Skotte et al. 2013) to
infer ancestry proportions in \emph{R. boylii} individuals. NGSadmix is
a robust admixture method that can be applied to low-depth NGS data, and
does not require called genotypes, thus reducing error associated with
potential ascertainment and uncertainty in the data (Skotte et al.
2013).

\hypertarget{genetic-differentiation-and-diversity-estimates-1}{%
\subsection{Genetic differentiation and diversity
estimates}\label{genetic-differentiation-and-diversity-estimates-1}}

\emph{Rana boylii} are cryptic, and often occur in low densities within
the study area. Thus, we retained a minimum of three individuals per
site for estimates of genetic diversity and F\textsubscript{ST}. With
genomic data, population genetic parameters can be accurately estimated
from even low sample numbers (Hotaling et al. 2018), and genomic
analyses in non-model organism often use fewer loci (Narum et al. 2013).
To quantify genetic variation and differentiation, pairwise population
differentiation (F\textsubscript{ST}) was calculated and scaled {[}mean
F\textsubscript{ST} / (1-mean F\textsubscript{ST}){]} to examine the
relationship between genetic differentiation and geographic distance
between populations (Wright 1943, Weir and Cockerham 1984, Rousset
1997). F\textsubscript{ST} was estimated by first calculating a folded
site frequency spectrum (SFS) for each population from site allele
frequencies (SAF) in ANGSD (doSaf 1, fold 1, minMapQ 10, minQ 20, GL 2)
and specifying the Rapture bait locations using the -sites flag (Nielsen
et al. 2012). The two-dimensional SFS between each population pair were
then estimated from folded SAF.idx files using a maxIter of 100 with
realSFS (Korneliussen et al. 2014). F\textsubscript{ST} statistics were
then calculated from two-dimensional SFS (2DSFS) for each possible
pairwise combination of unique collection locations using an estimator
preferable for small sample sizes implemented in ANGSD (-whichFST 1).
These values were plotted in R.

We summarized patterns of genetic variation using two two estimators of
\(\theta\) (\(4N\mu\)): Tajima's \(\theta\) (\(\theta_\pi\)) is based on
the average number of pairwise differences (Tajima 1983) and Watterson's
\(\theta\) (\(\theta_S\)) is based on the number of segregating sites
(Watterson 1975). These estimators are influenced by the demographic
history of a population and provide information on the trajectory of
changes in genetic diversity. When genetic diversity has been stable,
these estimates are generally equal; but when genetic diversity has been
increasing, \(\theta_\pi > \theta_S\); and when genetic diversity has
been decreasing, \(\theta_S > \theta_\pi\). To calculate \(\theta\)
statistics from Rapture data, we used folded SFS in ANGSD with -GL 2,
-doThetas 1, -doSaf 1 -fold 1, and -pest. Outputs were used to calculate
each statistic for each site using thetaStat with make\_bed and then
do\_stat. These data were averaged over the sites to obtain a single
``genome-wide'' value for each statistic for each locality (Korneliussen
et al. 2013).

\hypertarget{results-2}{%
\section{RESULTS}\label{results-2}}

A total of 1,103 individual samples were sequenced using Rapture (see
Methods). For principal components analysis (PCA) and admixture, we
selected samples that had greater than 100,000 alignments and had 1 or
more individuals per sampling locality. For localities with greater than
ten individuals, we randomly sampled a maximum of 10 samples, yielding
480 total samples from 89 distinct localities across the range of the
species (Figure 1, Table 1). These localities overlap many of the
localities used in McCartney-Melstad et al. (2018), with a few notable
differences. There were more individuals available for analyses at most
of the localities (Table 1), there was higher resolution sampling in
certain areas (i.e., the northern coast of California, the Feather
watershed), and a two additional localities fall outside of the clades
delineated by McCartney-Melstad et al. (2018) (i.e., Locality 1 in the
SF American basin in El Dorado County, and Locality 4 in the Honey-Eagle
Lakes basin in Lassen County; Figure 1).

\appendix

\hypertarget{the-first-appendix}{%
\chapter{The First Appendix}\label{the-first-appendix}}

This first appendix includes all of the R chunks of code that were
hidden throughout the document (using the \texttt{include\ =\ FALSE}
chunk tag) to help with readibility and/or setup.

\textbf{In the main Rmd file}
\begin{Shaded}
\begin{Highlighting}[]
\CommentTok{# This chunk ensures that the huskydown package is}
\CommentTok{# installed and loaded. This huskydown package includes}
\CommentTok{# the template files for the thesis.}
\ControlFlowTok{if}\NormalTok{(}\OperatorTok{!}\KeywordTok{require}\NormalTok{(devtools))}
  \KeywordTok{install.packages}\NormalTok{(}\StringTok{"devtools"}\NormalTok{, }\DataTypeTok{repos =} \StringTok{"http://cran.rstudio.com"}\NormalTok{)}
\ControlFlowTok{if}\NormalTok{(}\OperatorTok{!}\KeywordTok{require}\NormalTok{(huskydown))}
\NormalTok{  devtools}\OperatorTok{::}\KeywordTok{install_github}\NormalTok{(}\StringTok{"benmarwick/huskydown"}\NormalTok{)}
\KeywordTok{library}\NormalTok{(huskydown)}
\CommentTok{#if(!require(huskydown))}
\CommentTok{#  devtools::install_github("danovando/gauchodown")}
\CommentTok{#library(gauchodown)}
\KeywordTok{library}\NormalTok{(knitr)}
\end{Highlighting}
\end{Shaded}
\textbf{In Chapter \ref{ref-labels}:}

\hypertarget{the-second-appendix-for-fun}{%
\chapter{The Second Appendix, for
Fun}\label{the-second-appendix-for-fun}}

\hypertarget{colophon}{%
\chapter*{Colophon}\label{colophon}}
\addcontentsline{toc}{chapter}{Colophon}

This document is set in \href{https://github.com/georgd/EB-Garamond}{EB
Garamond}, \href{https://github.com/adobe-fonts/source-code-pro/}{Source
Code Pro} and \href{http://www.latofonts.com/lato-free-fonts/}{Lato}.
The body text is set at 11pt with \(\familydefault\).

It was written in R Markdown and \(\LaTeX\), and rendered into PDF using
\href{https://github.com/benmarwick/huskydown}{huskydown} and
\href{https://github.com/rstudio/bookdown}{bookdown}.

This document was typeset using the XeTeX typesetting system, and the
\href{http://staff.washington.edu/fox/tex/}{University of Washington
Thesis class} class created by Jim Fox. Under the hood, the
\href{https://github.com/UWIT-IAM/UWThesis}{University of Washington
Thesis LaTeX template} is used to ensure that documents conform
precisely to submission standards. Other elements of the document
formatting source code have been taken from the
\href{https://github.com/stevenpollack/ucbthesis}{Latex, Knitr, and
RMarkdown templates for UC Berkeley's graduate thesis}, and
\href{https://github.com/suchow/Dissertate}{Dissertate: a LaTeX
dissertation template to support the production and typesetting of a PhD
dissertation at Harvard, Princeton, and NYU}

The source files for this thesis, along with all the data files, have
been organised into an R package, xxx, which is available at
\url{https://github.com/xxx/xxx}. A hard copy of the thesis can be found
in the University of Washington library.

This version of the thesis was generated on 2018-09-10 17:50:58. The
repository is currently at this commit:

The computational environment that was used to generate this version is
as follows:
\begin{verbatim}
Session info -------------------------------------------------------------
\end{verbatim}
\begin{verbatim}
 setting  value                       
 version  R version 3.5.1 (2018-07-02)
 system   x86_64, darwin15.6.0        
 ui       X11                         
 language (EN)                        
 collate  en_US.UTF-8                 
 tz       America/Los_Angeles         
 date     2018-09-10                  
\end{verbatim}
\begin{verbatim}
Packages -----------------------------------------------------------------
\end{verbatim}
\begin{verbatim}
 package     * version    date       source                               
 assertthat    0.2.0      2017-04-11 CRAN (R 3.5.0)                       
 backports     1.1.2      2017-12-13 CRAN (R 3.5.0)                       
 base        * 3.5.1      2018-07-05 local                                
 bindr         0.1.1      2018-03-13 CRAN (R 3.5.0)                       
 bindrcpp      0.2.2      2018-03-29 CRAN (R 3.5.0)                       
 bookdown    * 0.7        2018-02-18 CRAN (R 3.5.0)                       
 broom         0.5.0      2018-07-17 CRAN (R 3.5.0)                       
 cellranger    1.1.0      2016-07-27 CRAN (R 3.5.0)                       
 cli           1.0.0      2017-11-05 CRAN (R 3.5.0)                       
 colorspace    1.3-2      2016-12-14 CRAN (R 3.5.0)                       
 compiler      3.5.1      2018-07-05 local                                
 crayon        1.3.4      2017-09-16 CRAN (R 3.5.0)                       
 datasets    * 3.5.1      2018-07-05 local                                
 devtools    * 1.13.6     2018-06-27 CRAN (R 3.5.0)                       
 digest        0.6.16     2018-08-22 CRAN (R 3.5.1)                       
 dplyr       * 0.7.6      2018-06-29 CRAN (R 3.5.0)                       
 evaluate      0.11       2018-07-17 CRAN (R 3.5.0)                       
 forcats     * 0.3.0      2018-02-19 CRAN (R 3.5.0)                       
 ggplot2     * 3.0.0.9000 2018-09-04 Github (tidyverse/ggplot2@6e545dc)   
 git2r         0.23.0     2018-07-17 CRAN (R 3.5.0)                       
 glue          1.3.0      2018-07-17 CRAN (R 3.5.0)                       
 graphics    * 3.5.1      2018-07-05 local                                
 grDevices   * 3.5.1      2018-07-05 local                                
 grid          3.5.1      2018-07-05 local                                
 gtable        0.2.0      2016-02-26 CRAN (R 3.5.0)                       
 haven         1.1.2      2018-06-27 CRAN (R 3.5.0)                       
 hms           0.4.2      2018-03-10 CRAN (R 3.5.0)                       
 htmltools     0.3.6      2017-04-28 CRAN (R 3.5.0)                       
 httr          1.3.1      2017-08-20 CRAN (R 3.5.0)                       
 huskydown   * 0.0.5      2018-09-04 Github (benmarwick/huskydown@3ef00c9)
 jsonlite      1.5        2017-06-01 CRAN (R 3.5.0)                       
 kableExtra  * 0.9.0      2018-05-21 CRAN (R 3.5.0)                       
 knitr       * 1.20       2018-02-20 CRAN (R 3.5.0)                       
 lattice       0.20-35    2017-03-25 CRAN (R 3.5.1)                       
 lazyeval      0.2.1      2017-10-29 CRAN (R 3.5.0)                       
 lubridate     1.7.4      2018-04-11 CRAN (R 3.5.0)                       
 magrittr      1.5        2014-11-22 CRAN (R 3.5.0)                       
 memoise       1.1.0      2017-04-21 CRAN (R 3.5.0)                       
 methods     * 3.5.1      2018-07-05 local                                
 modelr        0.1.2      2018-05-11 CRAN (R 3.5.0)                       
 munsell       0.5.0      2018-06-12 CRAN (R 3.5.0)                       
 nlme          3.1-137    2018-04-07 CRAN (R 3.5.1)                       
 pillar        1.3.0      2018-07-14 CRAN (R 3.5.0)                       
 pkgconfig     2.0.2      2018-08-16 CRAN (R 3.5.0)                       
 plyr          1.8.4      2016-06-08 CRAN (R 3.5.0)                       
 purrr       * 0.2.5      2018-05-29 CRAN (R 3.5.0)                       
 R6            2.2.2      2017-06-17 CRAN (R 3.5.0)                       
 Rcpp          0.12.18    2018-07-23 CRAN (R 3.5.1)                       
 readr       * 1.1.1      2017-05-16 CRAN (R 3.5.0)                       
 readxl        1.1.0      2018-04-20 CRAN (R 3.5.0)                       
 rlang         0.2.2      2018-08-16 CRAN (R 3.5.0)                       
 rmarkdown     1.10       2018-06-11 cran (@1.10)                         
 rprojroot     1.3-2      2018-01-03 CRAN (R 3.5.0)                       
 rstudioapi    0.7        2017-09-07 CRAN (R 3.5.0)                       
 rvest         0.3.2      2016-06-17 CRAN (R 3.5.0)                       
 scales        1.0.0.9000 2018-08-29 Github (hadley/scales@0f7a186)       
 stats       * 3.5.1      2018-07-05 local                                
 stringi       1.2.4      2018-07-20 CRAN (R 3.5.0)                       
 stringr     * 1.3.1      2018-05-10 CRAN (R 3.5.0)                       
 tibble      * 1.4.2      2018-01-22 CRAN (R 3.5.0)                       
 tidyr       * 0.8.1      2018-05-18 CRAN (R 3.5.0)                       
 tidyselect    0.2.4      2018-02-26 CRAN (R 3.5.0)                       
 tidyverse   * 1.2.1      2017-11-14 CRAN (R 3.5.0)                       
 tools         3.5.1      2018-07-05 local                                
 utils       * 3.5.1      2018-07-05 local                                
 viridisLite   0.3.0      2018-02-01 CRAN (R 3.5.0)                       
 withr         2.1.2      2018-08-29 Github (jimhester/withr@8b9cee2)     
 xfun          0.3        2018-07-06 CRAN (R 3.5.0)                       
 xml2          1.2.0      2018-01-24 CRAN (R 3.5.0)                       
 yaml          2.2.0      2018-07-25 CRAN (R 3.5.0)                       
\end{verbatim}
\backmatter

\hypertarget{references}{%
\chapter*{References}\label{references}}
\addcontentsline{toc}{chapter}{References}

\markboth{References}{References}

\noindent

\setlength{\parindent}{-0.20in}
\setlength{\leftskip}{0.20in}
\setlength{\parskip}{8pt}

\hypertarget{refs}{}
\leavevmode\hypertarget{ref-abbott_hybridization_2013}{}%
Abbott, R., D. Albach, S. Ansell, J. W. Arntzen, S. J. E. Baird, N.
Bierne, J. Boughman, A. Brelsford, C. A. Buerkle, R. Buggs, R. K.
Butlin, U. Dieckmann, F. Eroukhmanoff, A. Grill, S. H. Cahan, J. S.
Hermansen, G. Hewitt, A. G. Hudson, C. Jiggins, J. Jones, B. Keller, T.
Marczewski, J. Mallet, P. Martinez-Rodriguez, M. Möst, S. Mullen, R.
Nichols, A. W. Nolte, C. Parisod, K. Pfennig, A. M. Rice, M. G. Ritchie,
B. Seifert, C. M. Smadja, R. Stelkens, J. M. Szymura, R. Väinölä, J. B.
W. Wolf, and D. Zinner. 2013. Hybridization and speciation. Journal of
evolutionary biology 26:229--246.

\leavevmode\hypertarget{ref-ali_rad_2016}{}%
Ali, O. A., S. M. O'Rourke, S. J. Amish, M. H. Meek, G. Luikart, C.
Jeffres, and M. R. Miller. 2016. RAD Capture (Rapture): Flexible and
Efficient Sequence-Based Genotyping. Genetics 202:389--400.

\leavevmode\hypertarget{ref-anderson_oceanic_2009}{}%
Anderson, J. J., and W. N. Beer. 2009. Oceanic, riverine, and genetic
influences on spring chinook salmon migration timing. Ecological
applications: a publication of the Ecological Society of America
19:1989--2003.

\leavevmode\hypertarget{ref-baird_rapid_2008}{}%
Baird, N. A., P. D. Etter, T. S. Atwood, M. C. Currey, A. L. Shiver, Z.
A. Lewis, E. U. Selker, W. A. Cresko, and E. A. Johnson. 2008. Rapid SNP
discovery and genetic mapping using sequenced RAD markers. PloS one
3:e3376.

\leavevmode\hypertarget{ref-baird_descriptions_1856}{}%
Baird, S. F. 1856. Descriptions of new genera and species of North
American Frogs. Pages 59--62 \emph{in} Proceedings of the Academy of
Natural Sciences of Philadelphia. Philadelphia, Academy of Natural
Sciences of Philadelphia.

\leavevmode\hypertarget{ref-barbosa_integrative_2018}{}%
Barbosa, S., F. Mestre, T. A. White, J. Paupério, P. C. Alves, and J. B.
Searle. 2018. Integrative approaches to guide conservation decisions:
Using genomics to define conservation units and functional corridors.
Molecular ecology.

\leavevmode\hypertarget{ref-barrera-guzman_hybrid_2018}{}%
Barrera-Guzmán, A. O., A. Aleixo, M. D. Shawkey, and J. T. Weir. 2018.
Hybrid speciation leads to novel male secondary sexual ornamentation of
an Amazonian bird. Proceedings of the National Academy of Sciences of
the United States of America 115:E218--E225.

\leavevmode\hypertarget{ref-beebee_amphibian_2005}{}%
Beebee, T. J. C., and R. A. Griffiths. 2005. The amphibian decline
crisis: A watershed for conservation biology? Biological conservation
125:271--285.

\leavevmode\hypertarget{ref-broquet_buccal_2007}{}%
Broquet, T., L. Berset-Braendli, G. Emaresi, and L. Fumagalli. 2007.
Buccal swabs allow efficient and reliable microsatellite genotyping in
amphibians. Conservation genetics 8:509--511.

\leavevmode\hypertarget{ref-bunn_basic_2002}{}%
Bunn, S. E., and A. H. Arthington. 2002. Basic principles and ecological
consequences of altered flow regimes for aquatic biodiversity.
Environmental management 30:492--507.

\leavevmode\hypertarget{ref-burkey_extinction_1989}{}%
Burkey, T. V. 1989. Extinction in Nature Reserves - the Effect of
Fragmentation and the Importance of Migration between Reserve Fragments.
Oikos 55:75--81.

\leavevmode\hypertarget{ref-camp_notes_1917}{}%
Camp, C. L. 1917. Notes on the systematic status of the toads and frogs
of California. University of California publications in zoology
17:115--125.

\leavevmode\hypertarget{ref-caze_could_2016}{}%
Cazé, A. L. R., G. Mäder, T. S. Nunes, L. P. Queiroz, G. de Oliveira, J.
A. F. Diniz-Filho, S. L. Bonatto, and L. B. Freitas. 2016. Could refuge
theory and rivers acting as barriers explain the genetic variability
distribution in the Atlantic Forest? Molecular phylogenetics and
evolution 101:242--251.

\leavevmode\hypertarget{ref-davidson_declining_2004}{}%
Davidson, C. 2004. DECLINING DOWNWIND: AMPHIBIAN POPULATION DECLINES IN
CALIFORNIA AND HISTORICAL PESTICIDE USE. Ecological applications: a
publication of the Ecological Society of America 14:1892--1902.

\leavevmode\hypertarget{ref-davidson_spatial_2002}{}%
Davidson, C., H. B. Shaffer, M. R. Jennings - Conservation Biology, and
2002. 2002. Spatial tests of the pesticide drift, habitat destruction,
UV‐B, and climate‐change hypotheses for California amphibian declines.
Conservation biology: the journal of the Society for Conservation
Biology 16:1588--1601.

\leavevmode\hypertarget{ref-drost_collapse_1996}{}%
Drost, C. A., and G. M. Fellers. 1996. Collapse of a Regional Frog Fauna
in the Yosemite Area of the California Sierra Nevada, USA. Conservation
biology: the journal of the Society for Conservation Biology
10:414--425.

\leavevmode\hypertarget{ref-dudgeon_freshwater_2006}{}%
Dudgeon, D., A. H. Arthington, M. O. Gessner, Z.-I. Kawabata, D. J.
Knowler, C. Lévêque, R. J. Naiman, A.-H. Prieur-Richard, D. Soto, M. L.
J. Stiassny, and C. A. Sullivan. 2006. Freshwater biodiversity:
Importance, threats, status and conservation challenges. Biological
reviews of the Cambridge Philosophical Society 81:163--182.

\leavevmode\hypertarget{ref-fumagalli_quantifying_2013}{}%
Fumagalli, M., F. G. Vieira, T. S. Korneliussen, T. Linderoth, E.
Huerta-Sánchez, A. Albrechtsen, and R. Nielsen. 2013. Quantifying
population genetic differentiation from next-generation sequencing data.
Genetics 195:979--992.

\leavevmode\hypertarget{ref-fumagalli_ngstools:_2014}{}%
Fumagalli, M., F. G. Vieira, T. Linderoth, and R. Nielsen. 2014.
ngsTools: Methods for population genetics analyses from next-generation
sequencing data. Bioinformatics 30:1486--1487.

\leavevmode\hypertarget{ref-goldberg_frogs_2003}{}%
Goldberg, C. S., M. E. Kaplan, and C. R. Schwalbe. 2003. From the frog's
mouth: Buccal swabs for collection of DNA from amphibians.
Herpetological review 34:220--221.

\leavevmode\hypertarget{ref-hendricks_recent_2018}{}%
Hendricks, S., E. C. Anderson, T. Antao, L. Bernatchez, B. R. Forester,
B. Garner, B. K. Hand, P. A. Hohenlohe, M. Kardos, B. Koop, A.
Sethuraman, R. S. Waples, and G. Luikart. 2018. Recent advances in
conservation and population genomics data analysis. Evolutionary
applications.

\leavevmode\hypertarget{ref-heyer_measuring_1994}{}%
Heyer, W. R., M. A. Donnelly, R. W. McDiarmid, L.-A. C. Hayek, and M. S.
Foster. 1994. Measuring and monitoring biological diversity. Standard
methods for amphibians. Smithsonian Institution Press, Washington DC.

\leavevmode\hypertarget{ref-hotaling_demographic_2018}{}%
Hotaling, S., C. C. Muhlfeld, J. J. Giersch, O. A. Ali, S. Jordan, M. R.
Miller, G. Luikart, and D. W. Weisrock. 2018. Demographic modelling
reveals a history of divergence with gene flow for a glacially tied
stonefly in a changing post-Pleistocene landscape. Journal of
biogeography 45:304--317.

\leavevmode\hypertarget{ref-jennings_amphibian_1994}{}%
Jennings, M. R., and M. P. Hayes. 1994. Amphibian and reptile species of
special concern in California. Final Report. California Department of
Fish; Game Inland Fisheries Division, Rancho Cordova.

\leavevmode\hypertarget{ref-kim_estimation_2011}{}%
Kim, S. Y., K. E. Lohmueller, A. Albrechtsen, Y. Li, T. Korneliussen, G.
Tian, N. Grarup, T. Jiang, G. Andersen, D. Witte, T. Jorgensen, T.
Hansen, O. Pedersen, J. Wang, and R. Nielsen. 2011. Estimation of allele
frequency and association mapping using next-generation sequencing data.
BMC bioinformatics 12:231.

\leavevmode\hypertarget{ref-korneliussen_angsd:_2014}{}%
Korneliussen, T. S., A. Albrechtsen, and R. Nielsen. 2014. ANGSD:
Analysis of Next Generation Sequencing Data. BMC bioinformatics 15:356.

\leavevmode\hypertarget{ref-korneliussen_calculation_2013}{}%
Korneliussen, T. S., I. Moltke, A. Albrechtsen, and R. Nielsen. 2013.
Calculation of Tajima's D and other neutrality test statistics from low
depth next-generation sequencing data. BMC bioinformatics 14:289.

\leavevmode\hypertarget{ref-kupferberg_hydrologic_1996}{}%
Kupferberg, S. J. 1996. Hydrologic and geomorphic factors affecting
conservation of a river-breeding frog (Rana boylii). Ecological
applications: a publication of the Ecological Society of America
6:1332--1344.

\leavevmode\hypertarget{ref-kupferberg_effects_2012}{}%
Kupferberg, S. J., W. J. Palen, A. J. Lind, S. Bobzien, A. Catenazzi, J.
Drennan, and M. E. Power. 2012. Effects of flow regimes altered by dams
on survival, population declines, and range-wide losses of California
river-breeding frogs. Conservation biology: the journal of the Society
for Conservation Biology 26:513--524.

\leavevmode\hypertarget{ref-lande_role_1996}{}%
Lande, R., and S. Shannon. 1996. THE ROLE OF GENETIC VARIATION IN
ADAPTATION AND POPULATION PERSISTENCE IN A CHANGING ENVIRONMENT.
Evolution; international journal of organic evolution 50:434--437.

\leavevmode\hypertarget{ref-li_aligning_2013}{}%
Li, H. 2013. Aligning sequence reads, clone sequences and assembly
contigs with BWA-MEM.

\leavevmode\hypertarget{ref-li_fast_2010}{}%
Li, H., and R. Durbin. 2010. Fast and accurate long-read alignment with
Burrows-Wheeler transform. Bioinformatics 26:589--595.

\leavevmode\hypertarget{ref-li_inference_2011}{}%
Li, H., and R. Durbin. 2011. Inference of human population history from
individual whole-genome sequences. Nature 475:493--496.

\leavevmode\hypertarget{ref-li_sequence_2009}{}%
Li, H., B. Handsaker, A. Wysoker, T. Fennell, J. Ruan, N. Homer, G.
Marth, G. Abecasis, R. Durbin, and 1000 Genome Project Data Processing
Subgroup. 2009. The Sequence Alignment/Map format and SAMtools.
Bioinformatics 25:2078--2079.

\leavevmode\hypertarget{ref-li_ten_2017}{}%
Li, Y., X.-X. Zhang, R.-L. Mao, J. Yang, C.-Y. Miao, Z. Li, and Y.-X.
Qiu. 2017. Ten Years of Landscape Genomics: Challenges and
Opportunities. Frontiers in plant science 8:2136.

\leavevmode\hypertarget{ref-lind_effects_1996}{}%
Lind, A. J., H. H. Welsh Jr, and R. A. Wilson. 1996. The effects of a
dam on breeding habitat and egg survival of the foothill yellow-legged
frog (Rana boylii) in Northwestern Calfifornia. Herpetological review
27:62--66.

\leavevmode\hypertarget{ref-mallet_hybrid_2007}{}%
Mallet, J. 2007. Hybrid speciation. Nature 446:279--283.

\leavevmode\hypertarget{ref-mccartney-melstad_population_2018}{}%
McCartney-Melstad, E., M. Gidiş, and H. B. Shaffer. 2018. Population
genomic data reveal extreme geographic subdivision and novel
conservation actions for the declining foothill yellow-legged frog.
Heredity.

\leavevmode\hypertarget{ref-miller_conserved_2012}{}%
Miller, M. R., J. P. Brunelli, P. A. Wheeler, S. Liu, C. E. Rexroad 3rd,
Y. Palti, C. Q. Doe, and G. H. Thorgaard. 2012. A conserved haplotype
controls parallel adaptation in geographically distant salmonid
populations. Molecular ecology 21:237--249.

\leavevmode\hypertarget{ref-miller_rapid_2007}{}%
Miller, M. R., J. P. Dunham, A. Amores, W. A. Cresko, and E. A. Johnson.
2007. Rapid and cost-effective polymorphism identification and
genotyping using restriction site associated DNA (RAD) markers. Genome
research 17:240--248.

\leavevmode\hypertarget{ref-moyle_rapid_2011}{}%
Moyle, P. B., J. V. E. Katz, and R. M. Quiñones. 2011. Rapid decline of
California's native inland fishes: A status assessment. Biological
conservation 144:2414--2423.

\leavevmode\hypertarget{ref-murchie_fish_2008}{}%
Murchie, K. J., K. P. E. Hair, C. E. Pullen, T. D. Redpath, H. R.
Stephens, and S. J. Cooke. 2008. Fish response to modified flow regimes
in regulated rivers: Research methods, effects and opportunities. River
research and applications 24:197--217.

\leavevmode\hypertarget{ref-narum_genotyping-by-sequencing_2013}{}%
Narum, S. R., C. A. Buerkle, J. W. Davey, M. R. Miller, and P. A.
Hohenlohe. 2013. Genotyping-by-sequencing in ecological and conservation
genomics. Molecular ecology 22:2841--2847.

\leavevmode\hypertarget{ref-nielsen_snp_2012}{}%
Nielsen, R., T. Korneliussen, A. Albrechtsen, Y. Li, and J. Wang. 2012.
SNP calling, genotype calling, and sample allele frequency estimation
from New-Generation Sequencing data. PloS one 7:e37558.

\leavevmode\hypertarget{ref-nilsson_fragmentation_2005}{}%
Nilsson, C., C. A. Reidy, M. Dynesius, and C. Revenga. 2005.
Fragmentation and flow regulation of the world's large river systems.
Science 308:405--408.

\leavevmode\hypertarget{ref-nunziata_genomic_2017}{}%
Nunziata, S. O., S. L. Lance, D. E. Scott, E. M. Lemmon, and D. W.
Weisrock. 2017. Genomic data detect corresponding signatures of
population size change on an ecological time scale in two salamander
species. Molecular ecology 26:1060--1074.

\leavevmode\hypertarget{ref-parris_assessing_2010}{}%
Parris, K. M., S. C. McCall, M. A. McCarthy, B. A. Minteer, K. Steele,
S. Bekessy, and F. Medvecky. 2010. Assessing ethical trade-offs in
ecological field studies. The Journal of applied ecology 47:227--234.

\leavevmode\hypertarget{ref-peek_landscape_2010}{}%
Peek, R. 2010. Landscape Genetics of Foothill Yellow-Legged Frogs (Rana
boylii) in regulated and unregulated rivers: Assessing connectivity and
genetic fragmentation. PhD Thesis, University of San Francisco.

\leavevmode\hypertarget{ref-pidancier_buccal_2003}{}%
Pidancier, N., C. Miquel, and C. Miaud. 2003. Buccal swabs as a
non-destructive tissue sampling method for DNA analysis in amphibians.
The Herpetological journal 13:175--178.

\leavevmode\hypertarget{ref-poff_homogenization_2007}{}%
Poff, N. L., J. D. Olden, D. M. Merritt, and D. M. Pepin. 2007.
Homogenization of regional river dynamics by dams and global
biodiversity implications. Proceedings of the National Academy of
Sciences of the United States of America 104:5732--5737.

\leavevmode\hypertarget{ref-power_dams_1996}{}%
Power, M. E., W. E. Dietrich, and J. C. Finlay. 1996. Dams and
Downstream Aquatic Biodiversity: Potential Food Web Consequences of
Hydrologic and Geomorphic Change. Environmental management 20:887--895.

\leavevmode\hypertarget{ref-prince_evolutionary_2017}{}%
Prince, D. J., S. M. O'Rourke, T. Q. Thompson, O. A. Ali, H. S. Lyman,
I. K. Saglam, T. J. Hotaling, A. P. Spidle, and M. R. Miller. 2017. The
evolutionary basis of premature migration in Pacific salmon highlights
the utility of genomics for informing conservation. Science advances
3:e1603198.

\leavevmode\hypertarget{ref-pringle_what_2003}{}%
Pringle, C. 2003. What is hydrologic connectivity and why is it
ecologically important? Hydrological processes 17:2685--2689.

\leavevmode\hypertarget{ref-pringle_hydrologic_2001}{}%
Pringle, C. M. 2001. Hydrologic Connectivity and the Management of
Biological Reserves: A Global Perspective. Ecological applications: a
publication of the Ecological Society of America 11:981--998.

\leavevmode\hypertarget{ref-r_core_team_r:_2017}{}%
R Core Team. 2017. R: A Language and Environment for Statistical
Computing. R Foundation for Statistical Computing, Vienna, Austria.

\leavevmode\hypertarget{ref-ricciardi_extinction_1999}{}%
Ricciardi, A., and J. B. Rasmussen. 1999. Extinction rates of North
American freshwater fauna. Conservation biology: the journal of the
Society for Conservation Biology 13:1220--1222.

\leavevmode\hypertarget{ref-rousset_genetic_1997}{}%
Rousset, F. 1997. Genetic differentiation and estimation of gene flow
from F-statistics under isolation by distance. Genetics 145:1219--1228.

\leavevmode\hypertarget{ref-ruby_price:_2013}{}%
Ruby, J. G., P. Bellare, and J. L. Derisi. 2013. PRICE: Software for the
targeted assembly of components of (Meta) genomic sequence data. G3
3:865--880.

\leavevmode\hypertarget{ref-schick_directed_2007}{}%
Schick, R. S., and S. T. Lindley. 2007. Directed connectivity among fish
populations in a riverine network. The Journal of applied ecology
44:1116--1126.

\leavevmode\hypertarget{ref-shaffer_species_2004}{}%
Shaffer, H. B., G. M. Fellers, S. R. Voss, J. C. Oliver, and G. B.
Pauly. 2004. Species boundaries, phylogeography and conservation
genetics of the red-legged frog (Rana aurora/draytonii) complex.
Molecular ecology 13:2667--2677.

\leavevmode\hypertarget{ref-shaw_importance_2016}{}%
Shaw, E. A., E. Lange, J. D. Shucksmith, and D. N. Lerner. 2016.
Importance of partial barriers and temporal variation in flow when
modelling connectivity in fragmented river systems. Ecological
engineering 91:515--528.

\leavevmode\hypertarget{ref-skotte_estimating_2013}{}%
Skotte, L., T. S. Korneliussen, and A. Albrechtsen. 2013. Estimating
individual admixture proportions from next generation sequencing data.
Genetics 195:693--702.

\leavevmode\hypertarget{ref-stebbins_field_2003}{}%
Stebbins, R. C. 2003. A Field Guide to Western Reptiles and Amphibians.
3rd editions. Houghton Mifflin Harcourt, Boston.

\leavevmode\hypertarget{ref-tajima_evolutionary_1983}{}%
Tajima, F. 1983. Evolutionary relationship of DNA sequences in finite
populations. Genetics 105:437--460.

\leavevmode\hypertarget{ref-thomson_california_2016}{}%
Thomson, B., A. Wright, and B. Shaffer. 2016. California Amphibian and
Reptile Species of Special Concern. University of California Press,
Berkeley, CA., http://arssc.ucdavis.edu.

\leavevmode\hypertarget{ref-tonkin_seasonality_2017}{}%
Tonkin, J. D., M. T. Bogan, N. Bonada, B. Rios-Touma, and D. A. Lytle.
2017. Seasonality and predictability shape temporal species diversity.
Ecology 98:1201--1216.

\leavevmode\hypertarget{ref-usfws_endangered_2014}{}%
USFWS. 2014. Endangered and Threatened Wildlife and Plants; Endangered
Species Status for Sierra Nevada Yellow-Legged Frog and Northern
Distinct Population Segment of the Mountain Yellow-Legged Frog, and
Threatened Species Status for Yosemite Toad. Federal register
79:24255--24310.

\leavevmode\hypertarget{ref-voelker_river_2013}{}%
Voelker, G., B. D. Marks, C. Kahindo, U. A'genonga, F. Bapeamoni, L. E.
Duffie, J. W. Huntley, E. Mulotwa, S. A. Rosenbaum, and J. E. Light.
2013. River barriers and cryptic biodiversity in an evolutionary museum.
Ecology and evolution 3:536--545.

\leavevmode\hypertarget{ref-watterson_number_1975}{}%
Watterson, G. A. 1975. Number of Segregating Sites in Genetic Models
without Recombination. Theoretical population biology 7:256--276.

\leavevmode\hypertarget{ref-weir_estimating_1984}{}%
Weir, B. S., and C. C. Cockerham. 1984. Estimating F-Statistics for the
Analysis of Population-Structure. Evolution; international journal of
organic evolution 38:1358--1370.

\leavevmode\hypertarget{ref-wiens_riverine_2002}{}%
Wiens, J. A. 2002. Riverine landscapes: Taking landscape ecology into
the water. Freshwater biology 47:501--515.

\leavevmode\hypertarget{ref-wilbur_ecological_1990}{}%
Wilbur, H. M., and R. D. Semlitsch. 1990. Ecological Consequences of
Tail Injury in Rana Tadpoles. Copeia 1990:18--24.

\leavevmode\hypertarget{ref-wright_isolation_1943}{}%
Wright, S. 1943. Isolation by Distance. Genetics 28:114--38.

\leavevmode\hypertarget{ref-yarnell_ecology_2010}{}%
Yarnell, S. M., J. H. Viers, and J. F. Mount. 2010. Ecology and
Management of the Spring Snowmelt Recession. Bioscience 60:114--127.

\leavevmode\hypertarget{ref-zweifel_ecology_1955}{}%
Zweifel, R. G. 1955. Ecology, distribution, and systematics of frogs of
the Rana boylei group. University of California Publications in Zoology
54:207--291.

\end{ucmainmatter}
\end{document}

%---Set Headers and Footers ------------------------------------------------------
\pagestyle{fancy}
% UCD
%\renewcommand{\chaptermark}[1]{\markboth{{\sf #1 \hspace*{\fill} Chapter~}}{} }
% GAUCHO
\fancyhf{}
\renewcommand{\chaptermark}[1]{\markboth{{\sf #1 \hspace*{\fill} Chapter~\thechapter}}{} }
\renewcommand{\sectionmark}[1]{\markright{ {\sf Section~\thesection 
\hspace*{\fill} #1 }}}


\makeatletter \if@twoside \fancyhead[LO]{\small \rightmark} \fancyhead[RE]{\small\leftmark} \else \fancyhead[LO]{\small\leftmark}
\fancyhead[RE]{\small\rightmark} \fi

\def\cleardoublepage{\clearpage\if@openright \ifodd\c@page\else
  \hbox{}
  \vspace*{\fill}
  \begin{center}
    This page intentionally left blank
  \end{center}
  \vspace{\fill}
  \thispagestyle{plain}
  \newpage
  \fi \fi}
\makeatother
\fancyfoot[c]{\textrm{\textup{\thepage}}} % page number
\fancyfoot[C]{\thepage}
\renewcommand{\headrulewidth}{0.4pt}

\fancypagestyle{plain} { \fancyhf{} \fancyfoot[C]{\thepage}
\renewcommand{\headrulewidth}{0pt}
\renewcommand{\footrulewidth}{0pt}}
