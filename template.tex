%%%---PREAMBLE---%%%%%%%%%%%%%%%%%%%%%%%%%%%%
\documentclass[proquest,12pt,final]{ucthesis-CA2012} % change this for oneside, twoside, 11pt, draft, etc.

% fix for pandoc 1.14
\providecommand{\tightlist}{%
  \setlength{\itemsep}{0pt}\setlength{\parskip}{0pt}}

%--- Packages --------------------------------------------------------
\usepackage[lofdepth,lotdepth,caption=false]{subfig}
\usepackage{fancyhdr}
\usepackage{amsmath, amssymb, graphicx}
\usepackage{xspace}
\usepackage{braket}
\usepackage{color}
\usepackage{setspace}
\usepackage{fancyvrb}
\usepackage{array}
\usepackage{ifxetex,ifluatex}
\usepackage{etoolbox}

%% for the per mil symbol
\usepackage[nointegrals]{wasysym}

%% for more attractive tables
\usepackage{booktabs}
\usepackage{xcolor}
\usepackage{longtable}
\usepackage{lscape}
\usepackage{tabularx}

\usepackage[nostamp]{draftwatermark}
% % Use the following to make modification
\SetWatermarkText{DRAFT}
\SetWatermarkLightness{0.95}

% suppress bottom page numbers on 1st page of each chapt.
% because they overlap with text
%\patchcmd{\chapter}{plain}{empty}{}{} % turn off for UCD requirements of page number on every page

%---New Definitions and Commands------------------------------------------------------

\newtheorem{theorem}{Jibberish}

\bibliography{references}

\hyphenation{mar-gin-al-ia}

% from uw_template.tex

% commands and environments needed by pandoc snippets
% extracted from the output of `pandoc -s`
%% Make R markdown code chunks work

\ifxetex
  \usepackage{fontspec,xltxtra,xunicode}
  \defaultfontfeatures{Mapping=tex-text,Scale=MatchLowercase}
\else
  \ifluatex
    \usepackage{fontspec}
    \defaultfontfeatures{Mapping=tex-text,Scale=MatchLowercase}
  \else
    \usepackage[utf8]{inputenc}
  \fi
\fi
\DefineShortVerb[commandchars=\\\{\}]{\|}
\DefineVerbatimEnvironment{Highlighting}{Verbatim}{commandchars=\\\{\}}
% Add ',fontsize=\small' for more characters per line
\newenvironment{Shaded}{}{}
\newcommand{\KeywordTok}[1]{\textcolor[rgb]{0.00,0.44,0.13}{\textbf{{#1}}}}
\newcommand{\DataTypeTok}[1]{\textcolor[rgb]{0.56,0.13,0.00}{{#1}}}
\newcommand{\DecValTok}[1]{\textcolor[rgb]{0.25,0.63,0.44}{{#1}}}
\newcommand{\BaseNTok}[1]{\textcolor[rgb]{0.25,0.63,0.44}{{#1}}}
\newcommand{\FloatTok}[1]{\textcolor[rgb]{0.25,0.63,0.44}{{#1}}}
\newcommand{\CharTok}[1]{\textcolor[rgb]{0.25,0.44,0.63}{{#1}}}
\newcommand{\StringTok}[1]{\textcolor[rgb]{0.25,0.44,0.63}{{#1}}}
\newcommand{\CommentTok}[1]{\textcolor[rgb]{0.38,0.63,0.69}{\textit{{#1}}}}
\newcommand{\OtherTok}[1]{\textcolor[rgb]{0.00,0.44,0.13}{{#1}}}
\newcommand{\AlertTok}[1]{\textcolor[rgb]{1.00,0.00,0.00}{\textbf{{#1}}}}
\newcommand{\FunctionTok}[1]{\textcolor[rgb]{0.02,0.16,0.49}{{#1}}}
\newcommand{\RegionMarkerTok}[1]{{#1}}
\newcommand{\ErrorTok}[1]{\textcolor[rgb]{1.00,0.00,0.00}{\textbf{{#1}}}}
\newcommand{\NormalTok}[1]{{#1}}
\newcommand{\OperatorTok}[1]{\textcolor[rgb]{0.00,0.44,0.13}{\textbf{{#1}}}}
\newcommand{\BuiltInTok}[1]{\textcolor[rgb]{0.00,0.44,0.13}{\textbf{{#1}}}}
\newcommand{\ControlFlowTok}[1]{\textcolor[rgb]{0.00,0.44,0.13}{\textbf{{#1}}}}

\ifxetex
  \usepackage[setpagesize=false, % page size defined by xetex
              unicode=false, % unicode breaks when used with xetex
              xetex,
              colorlinks=true,
              linkcolor=blue]{hyperref}
\else
  \usepackage[unicode=true,
              colorlinks=true,
              linkcolor=blue]{hyperref}
\fi
\hypersetup{breaklinks=true, pdfborder={0 0 0}}
\setlength{\parindent}{0pt}
\setlength{\parskip}{6pt plus 2pt minus 1pt}
\setlength{\emergencystretch}{3em}  % prevent overfull lines
\setcounter{secnumdepth}{0}

%---Set Margins ------------------------------------------------------
\setlength\oddsidemargin{0.25 in} \setlength\evensidemargin{0.25 in} \setlength\textwidth{6.25 in} \setlength\textheight{8.5 in} %8.5
\setlength\footskip{0.25 in} \setlength\topmargin{0 in} \setlength\headheight{0.25 in} \setlength\headsep{0.25 in}

%%%---DOCUMENT---%%%%%%%%%%%%%%%%%%%%%%%%%%%%
\begin{document}


%=== Preliminary Pages ============================================
\begin{ucfrontmatter}

  %%%%%%%%%%%%%%%%%%%%%%%%%%%
  % TITLE PAGE INFORMATION % % modified to meet UCDavis
  %%%%%%%%%%%%%%%%%%%%%%%%%%%

  \title{$title$}
  \author{$author$}
  \prevdegreeA{$prevdegreeA$}
  \prevdegreeB{$prevdegreeB$}
  \report{$report$} 
  \degree{$degree$} 
  \degreemonth{$month$} \degreeyear{$year$}
  \chair{$chair$}  % this is your advisor
  \othermemberA{$signature1$} % This is a member of your committee
  \othermemberB{$signature2$} % This is a member of your committee
  \othermemberC{$signature3$} % This is a member of your committee
  \numberofmembers{3} % should match the number of entries above (chair + othermembers)
  \field{$program$}
  \campus{$uccampus$}
	
  % MAKE THE TITLE PAGE
	\maketitle
	
	% MAKE THE APPROVAL SHEET
	% \approvalpage

  % COPYRIGHT
  % \copyrightpage % if you want
  
  % DEDICATION %
  %%%%%%%%%%%%%%%%%%%%

    \begin{dedication}

      \vspace*{20ex} % was 25

      \begin{center}
      \begin{large} % was Large

        $dedication$

      \end{large}

      \end{center}

  \end{dedication}

  % ACKNOWLEDGEMENTS %
  %%%%%%%%%%%%%%%%%%%%
  \begin{acknowledgements}
    $acknowledgments$
  \end{acknowledgements}

  % removed CV section from this but see gauchodown or huskydown

  % ABSTRACT %
  %%%%%%%%%%%%%%%%%%%%%%%%%%%
  \begin{abstract}
    \addcontentsline{toc}{chapter}{Abstract}

    $abstract$

    %\abstractsignature
  \end{abstract}
  
  % Table of Contents %
  %%%%%%%%%%%%%%%%%%%%%%%%%%%
	\tableofcontents

\end{ucfrontmatter} % end of the preliminary pages

\begin{ucmainmatter}

$body$

\end{ucmainmatter}

\end{document}


%---Set Headers and Footers ------------------------------------------------------

% GAUCHO
\pagestyle{fancy}
%\renewcommand{\chaptermark}[1]{\markboth{{\sf #1 \hspace*{\fill} Chapter~\thechapter}}{} }
%\renewcommand{\chaptermark}[1]{\markboth{{\sf #1 \hspace*{\fill} Chapter~\thesection}}{} }
%\renewcommand{\sectionmark}[1]{\markright{ {\sf #1 \hspace*{\fill} Chapter~}}{} }
% \renewcommand{\sectionmark}[1]{\markright{ {\sf Section~\thesection \hspace*{\fill} #1 }}}
%\fancyhf{}

%\makeatletter \if@twoside \fancyhead[LO]{\small \rightmark} \fancyhead[RE]{\small\leftmark} \else \fancyhead[LO]{\small\leftmark}
%\fancyhead[RE]{\small\rightmark} \fi

\def\cleardoublepage{\clearpage\if@openright \ifodd\c@page\else
  \hbox{}
  \vspace*{\fill}
  \begin{center}
    This page intentionally left blank
  \end{center}
  \vspace{\fill}
  \thispagestyle{plain}
  \newpage
  \fi \fi}

\makeatother
\fancyfoot[c]{\textrm{\textup{\thepage}}} % page number
\fancyfoot[C]{\thepage}
\renewcommand{\headrulewidth}{0.4pt}

\fancypagestyle{plain} { \fancyhf{} \fancyfoot[C]{\thepage}
\renewcommand{\headrulewidth}{0pt}
\renewcommand{\footrulewidth}{0pt}}
